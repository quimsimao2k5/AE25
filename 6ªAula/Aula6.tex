% Options for packages loaded elsewhere
% Options for packages loaded elsewhere
\PassOptionsToPackage{unicode}{hyperref}
\PassOptionsToPackage{hyphens}{url}
\PassOptionsToPackage{dvipsnames,svgnames,x11names}{xcolor}
%
\documentclass[
  portuguese,
  letterpaper,
  DIV=11,
  numbers=noendperiod]{scrartcl}
\usepackage{xcolor}
\usepackage{amsmath,amssymb}
\setcounter{secnumdepth}{-\maxdimen} % remove section numbering
\usepackage{iftex}
\ifPDFTeX
  \usepackage[T1]{fontenc}
  \usepackage[utf8]{inputenc}
  \usepackage{textcomp} % provide euro and other symbols
\else % if luatex or xetex
  \usepackage{unicode-math} % this also loads fontspec
  \defaultfontfeatures{Scale=MatchLowercase}
  \defaultfontfeatures[\rmfamily]{Ligatures=TeX,Scale=1}
\fi
\usepackage{lmodern}
\ifPDFTeX\else
  % xetex/luatex font selection
\fi
% Use upquote if available, for straight quotes in verbatim environments
\IfFileExists{upquote.sty}{\usepackage{upquote}}{}
\IfFileExists{microtype.sty}{% use microtype if available
  \usepackage[]{microtype}
  \UseMicrotypeSet[protrusion]{basicmath} % disable protrusion for tt fonts
}{}
\makeatletter
\@ifundefined{KOMAClassName}{% if non-KOMA class
  \IfFileExists{parskip.sty}{%
    \usepackage{parskip}
  }{% else
    \setlength{\parindent}{0pt}
    \setlength{\parskip}{6pt plus 2pt minus 1pt}}
}{% if KOMA class
  \KOMAoptions{parskip=half}}
\makeatother
% Make \paragraph and \subparagraph free-standing
\makeatletter
\ifx\paragraph\undefined\else
  \let\oldparagraph\paragraph
  \renewcommand{\paragraph}{
    \@ifstar
      \xxxParagraphStar
      \xxxParagraphNoStar
  }
  \newcommand{\xxxParagraphStar}[1]{\oldparagraph*{#1}\mbox{}}
  \newcommand{\xxxParagraphNoStar}[1]{\oldparagraph{#1}\mbox{}}
\fi
\ifx\subparagraph\undefined\else
  \let\oldsubparagraph\subparagraph
  \renewcommand{\subparagraph}{
    \@ifstar
      \xxxSubParagraphStar
      \xxxSubParagraphNoStar
  }
  \newcommand{\xxxSubParagraphStar}[1]{\oldsubparagraph*{#1}\mbox{}}
  \newcommand{\xxxSubParagraphNoStar}[1]{\oldsubparagraph{#1}\mbox{}}
\fi
\makeatother

\usepackage{color}
\usepackage{fancyvrb}
\newcommand{\VerbBar}{|}
\newcommand{\VERB}{\Verb[commandchars=\\\{\}]}
\DefineVerbatimEnvironment{Highlighting}{Verbatim}{commandchars=\\\{\}}
% Add ',fontsize=\small' for more characters per line
\usepackage{framed}
\definecolor{shadecolor}{RGB}{241,243,245}
\newenvironment{Shaded}{\begin{snugshade}}{\end{snugshade}}
\newcommand{\AlertTok}[1]{\textcolor[rgb]{0.68,0.00,0.00}{#1}}
\newcommand{\AnnotationTok}[1]{\textcolor[rgb]{0.37,0.37,0.37}{#1}}
\newcommand{\AttributeTok}[1]{\textcolor[rgb]{0.40,0.45,0.13}{#1}}
\newcommand{\BaseNTok}[1]{\textcolor[rgb]{0.68,0.00,0.00}{#1}}
\newcommand{\BuiltInTok}[1]{\textcolor[rgb]{0.00,0.23,0.31}{#1}}
\newcommand{\CharTok}[1]{\textcolor[rgb]{0.13,0.47,0.30}{#1}}
\newcommand{\CommentTok}[1]{\textcolor[rgb]{0.37,0.37,0.37}{#1}}
\newcommand{\CommentVarTok}[1]{\textcolor[rgb]{0.37,0.37,0.37}{\textit{#1}}}
\newcommand{\ConstantTok}[1]{\textcolor[rgb]{0.56,0.35,0.01}{#1}}
\newcommand{\ControlFlowTok}[1]{\textcolor[rgb]{0.00,0.23,0.31}{\textbf{#1}}}
\newcommand{\DataTypeTok}[1]{\textcolor[rgb]{0.68,0.00,0.00}{#1}}
\newcommand{\DecValTok}[1]{\textcolor[rgb]{0.68,0.00,0.00}{#1}}
\newcommand{\DocumentationTok}[1]{\textcolor[rgb]{0.37,0.37,0.37}{\textit{#1}}}
\newcommand{\ErrorTok}[1]{\textcolor[rgb]{0.68,0.00,0.00}{#1}}
\newcommand{\ExtensionTok}[1]{\textcolor[rgb]{0.00,0.23,0.31}{#1}}
\newcommand{\FloatTok}[1]{\textcolor[rgb]{0.68,0.00,0.00}{#1}}
\newcommand{\FunctionTok}[1]{\textcolor[rgb]{0.28,0.35,0.67}{#1}}
\newcommand{\ImportTok}[1]{\textcolor[rgb]{0.00,0.46,0.62}{#1}}
\newcommand{\InformationTok}[1]{\textcolor[rgb]{0.37,0.37,0.37}{#1}}
\newcommand{\KeywordTok}[1]{\textcolor[rgb]{0.00,0.23,0.31}{\textbf{#1}}}
\newcommand{\NormalTok}[1]{\textcolor[rgb]{0.00,0.23,0.31}{#1}}
\newcommand{\OperatorTok}[1]{\textcolor[rgb]{0.37,0.37,0.37}{#1}}
\newcommand{\OtherTok}[1]{\textcolor[rgb]{0.00,0.23,0.31}{#1}}
\newcommand{\PreprocessorTok}[1]{\textcolor[rgb]{0.68,0.00,0.00}{#1}}
\newcommand{\RegionMarkerTok}[1]{\textcolor[rgb]{0.00,0.23,0.31}{#1}}
\newcommand{\SpecialCharTok}[1]{\textcolor[rgb]{0.37,0.37,0.37}{#1}}
\newcommand{\SpecialStringTok}[1]{\textcolor[rgb]{0.13,0.47,0.30}{#1}}
\newcommand{\StringTok}[1]{\textcolor[rgb]{0.13,0.47,0.30}{#1}}
\newcommand{\VariableTok}[1]{\textcolor[rgb]{0.07,0.07,0.07}{#1}}
\newcommand{\VerbatimStringTok}[1]{\textcolor[rgb]{0.13,0.47,0.30}{#1}}
\newcommand{\WarningTok}[1]{\textcolor[rgb]{0.37,0.37,0.37}{\textit{#1}}}

\usepackage{longtable,booktabs,array}
\usepackage{calc} % for calculating minipage widths
% Correct order of tables after \paragraph or \subparagraph
\usepackage{etoolbox}
\makeatletter
\patchcmd\longtable{\par}{\if@noskipsec\mbox{}\fi\par}{}{}
\makeatother
% Allow footnotes in longtable head/foot
\IfFileExists{footnotehyper.sty}{\usepackage{footnotehyper}}{\usepackage{footnote}}
\makesavenoteenv{longtable}
\usepackage{graphicx}
\makeatletter
\newsavebox\pandoc@box
\newcommand*\pandocbounded[1]{% scales image to fit in text height/width
  \sbox\pandoc@box{#1}%
  \Gscale@div\@tempa{\textheight}{\dimexpr\ht\pandoc@box+\dp\pandoc@box\relax}%
  \Gscale@div\@tempb{\linewidth}{\wd\pandoc@box}%
  \ifdim\@tempb\p@<\@tempa\p@\let\@tempa\@tempb\fi% select the smaller of both
  \ifdim\@tempa\p@<\p@\scalebox{\@tempa}{\usebox\pandoc@box}%
  \else\usebox{\pandoc@box}%
  \fi%
}
% Set default figure placement to htbp
\def\fps@figure{htbp}
\makeatother



\ifLuaTeX
\usepackage[bidi=basic]{babel}
\else
\usepackage[bidi=default]{babel}
\fi
% get rid of language-specific shorthands (see #6817):
\let\LanguageShortHands\languageshorthands
\def\languageshorthands#1{}


\setlength{\emergencystretch}{3em} % prevent overfull lines

\providecommand{\tightlist}{%
  \setlength{\itemsep}{0pt}\setlength{\parskip}{0pt}}



 


\KOMAoption{captions}{tableheading}
\makeatletter
\@ifpackageloaded{caption}{}{\usepackage{caption}}
\AtBeginDocument{%
\ifdefined\contentsname
  \renewcommand*\contentsname{Índice}
\else
  \newcommand\contentsname{Índice}
\fi
\ifdefined\listfigurename
  \renewcommand*\listfigurename{Lista de Figuras}
\else
  \newcommand\listfigurename{Lista de Figuras}
\fi
\ifdefined\listtablename
  \renewcommand*\listtablename{Lista de Tabelas}
\else
  \newcommand\listtablename{Lista de Tabelas}
\fi
\ifdefined\figurename
  \renewcommand*\figurename{Figura}
\else
  \newcommand\figurename{Figura}
\fi
\ifdefined\tablename
  \renewcommand*\tablename{Tabela}
\else
  \newcommand\tablename{Tabela}
\fi
}
\@ifpackageloaded{float}{}{\usepackage{float}}
\floatstyle{ruled}
\@ifundefined{c@chapter}{\newfloat{codelisting}{h}{lop}}{\newfloat{codelisting}{h}{lop}[chapter]}
\floatname{codelisting}{Listagem}
\newcommand*\listoflistings{\listof{codelisting}{Lista de Listagens}}
\makeatother
\makeatletter
\makeatother
\makeatletter
\@ifpackageloaded{caption}{}{\usepackage{caption}}
\@ifpackageloaded{subcaption}{}{\usepackage{subcaption}}
\makeatother
\usepackage{bookmark}
\IfFileExists{xurl.sty}{\usepackage{xurl}}{} % add URL line breaks if available
\urlstyle{same}
\hypersetup{
  pdftitle={Métodos de Regularização em Regressão: Ridge e Lasso},
  pdfauthor={Soraia Pereira},
  pdflang={pt},
  colorlinks=true,
  linkcolor={blue},
  filecolor={Maroon},
  citecolor={Blue},
  urlcolor={Blue},
  pdfcreator={LaTeX via pandoc}}


\title{Métodos de Regularização em Regressão: Ridge e Lasso}
\author{Soraia Pereira}
\date{2026-12-10}
\begin{document}
\maketitle


\section{Implementação em R
(pacotes)}\label{implementauxe7uxe3o-em-r-pacotes}

\begin{Shaded}
\begin{Highlighting}[]
\CommentTok{\# instalar se necessário}
\NormalTok{pkgs }\OtherTok{\textless{}{-}} \FunctionTok{c}\NormalTok{(}\StringTok{"glmnet"}\NormalTok{,}\StringTok{"tidyverse"}\NormalTok{,}\StringTok{"broom"}\NormalTok{)}
\NormalTok{new }\OtherTok{\textless{}{-}} \FunctionTok{setdiff}\NormalTok{(pkgs, }\FunctionTok{rownames}\NormalTok{(}\FunctionTok{installed.packages}\NormalTok{()))}
\ControlFlowTok{if}\NormalTok{(}\FunctionTok{length}\NormalTok{(new)) }\FunctionTok{install.packages}\NormalTok{(new)}
\FunctionTok{library}\NormalTok{(glmnet); }\FunctionTok{library}\NormalTok{(tidyverse); }\FunctionTok{library}\NormalTok{(broom)}
\end{Highlighting}
\end{Shaded}

\section{\texorpdfstring{Exemplo 1:
\texttt{mtcars}}{Exemplo 1: mtcars}}\label{exemplo-1-mtcars}

Vamos prever \texttt{mpg} a partir de \texttt{disp}, \texttt{hp},
\texttt{wt}, \texttt{qsec}, \texttt{am}, \texttt{gear}.

\begin{Shaded}
\begin{Highlighting}[]
\FunctionTok{set.seed}\NormalTok{(}\DecValTok{123}\NormalTok{)}
\NormalTok{D }\OtherTok{\textless{}{-}}\NormalTok{ mtcars }\SpecialCharTok{|\textgreater{}} \FunctionTok{mutate}\NormalTok{(}\FunctionTok{across}\NormalTok{(}\FunctionTok{c}\NormalTok{(am, gear), factor))}
\CommentTok{\# Matriz modelo (one{-}hot sem intercepto)}
\NormalTok{X }\OtherTok{\textless{}{-}} \FunctionTok{model.matrix}\NormalTok{(mpg }\SpecialCharTok{\textasciitilde{}}\NormalTok{ disp }\SpecialCharTok{+}\NormalTok{ hp }\SpecialCharTok{+}\NormalTok{ wt }\SpecialCharTok{+}\NormalTok{ qsec }\SpecialCharTok{+}\NormalTok{ am }\SpecialCharTok{+}\NormalTok{ gear, }\AttributeTok{data =}\NormalTok{ D)[, }\SpecialCharTok{{-}}\DecValTok{1}\NormalTok{]}
\NormalTok{y }\OtherTok{\textless{}{-}}\NormalTok{ D}\SpecialCharTok{$}\NormalTok{mpg}
\CommentTok{\# grid de lambdas padrão do glmnet}
\NormalTok{fit\_ridge }\OtherTok{\textless{}{-}} \FunctionTok{glmnet}\NormalTok{(X, y, }\AttributeTok{alpha =} \DecValTok{0}\NormalTok{)   }\CommentTok{\# ridge}
\NormalTok{fit\_lasso }\OtherTok{\textless{}{-}} \FunctionTok{glmnet}\NormalTok{(X, y, }\AttributeTok{alpha =} \DecValTok{1}\NormalTok{)   }\CommentTok{\# lasso}
\NormalTok{fit\_ridge; fit\_lasso}
\end{Highlighting}
\end{Shaded}

\begin{verbatim}

Call:  glmnet(x = X, y = y, alpha = 0) 

    Df  %Dev Lambda
1    7  0.00 5147.0
2    7  0.74 4690.0
3    7  0.81 4273.0
4    7  0.89 3894.0
5    7  0.98 3548.0
6    7  1.07 3232.0
7    7  1.18 2945.0
8    7  1.29 2684.0
9    7  1.41 2445.0
10   7  1.55 2228.0
11   7  1.70 2030.0
12   7  1.86 1850.0
13   7  2.04 1685.0
14   7  2.23 1536.0
15   7  2.45 1399.0
16   7  2.68 1275.0
17   7  2.93 1162.0
18   7  3.21 1058.0
19   7  3.52  964.5
20   7  3.85  878.8
21   7  4.21  800.7
22   7  4.60  729.6
23   7  5.03  664.8
24   7  5.49  605.7
25   7  6.00  551.9
26   7  6.54  502.9
27   7  7.14  458.2
28   7  7.78  417.5
29   7  8.48  380.4
30   7  9.24  346.6
31   7 10.05  315.8
32   7 10.93  287.8
33   7 11.88  262.2
34   7 12.89  238.9
35   7 13.98  217.7
36   7 15.15  198.3
37   7 16.39  180.7
38   7 17.72  164.7
39   7 19.13  150.0
40   7 20.62  136.7
41   7 22.20  124.6
42   7 23.86  113.5
43   7 25.61  103.4
44   7 27.43   94.2
45   7 29.34   85.9
46   7 31.31   78.2
47   7 33.35   71.3
48   7 35.45   65.0
49   7 37.60   59.2
50   7 39.79   53.9
51   7 42.02   49.1
52   7 44.26   44.8
53   7 46.51   40.8
54   7 48.75   37.2
55   7 50.98   33.9
56   7 53.17   30.9
57   7 55.33   28.1
58   7 57.42   25.6
59   7 59.46   23.3
60   7 61.42   21.3
61   7 63.29   19.4
62   7 65.07   17.7
63   7 66.76   16.1
64   7 68.35   14.7
65   7 69.84   13.4
66   7 71.22   12.2
67   7 72.51   11.1
68   7 73.69   10.1
69   7 74.78    9.2
70   7 75.78    8.4
71   7 76.69    7.6
72   7 77.51    7.0
73   7 78.26    6.3
74   7 78.94    5.8
75   7 79.56    5.3
76   7 80.12    4.8
77   7 80.62    4.4
78   7 81.08    4.0
79   7 81.49    3.6
80   7 81.86    3.3
81   7 82.21    3.0
82   7 82.52    2.7
83   7 82.80    2.5
84   7 83.06    2.3
85   7 83.30    2.1
86   7 83.52    1.9
87   7 83.73    1.7
88   7 83.92    1.6
89   7 84.09    1.4
90   7 84.26    1.3
91   7 84.42    1.2
92   7 84.57    1.1
93   7 84.71    1.0
94   7 84.84    0.9
95   7 84.97    0.8
96   7 85.09    0.7
97   7 85.20    0.7
98   7 85.31    0.6
99   7 85.42    0.6
100  7 85.52    0.5
\end{verbatim}

\begin{verbatim}

Call:  glmnet(x = X, y = y, alpha = 1) 

   Df  %Dev Lambda
1   0  0.00 5.1470
2   1 12.78 4.6900
3   1 23.39 4.2730
4   2 32.47 3.8940
5   2 40.22 3.5480
6   3 46.85 3.2320
7   3 52.93 2.9450
8   3 57.98 2.6840
9   3 62.18 2.4450
10  3 65.66 2.2280
11  3 68.55 2.0300
12  3 70.95 1.8500
13  3 72.94 1.6850
14  3 74.60 1.5360
15  3 75.97 1.3990
16  3 77.11 1.2750
17  3 78.05 1.1620
18  3 78.84 1.0580
19  4 79.61 0.9645
20  4 80.35 0.8788
21  4 80.98 0.8007
22  5 81.54 0.7296
23  4 82.30 0.6648
24  4 82.89 0.6057
25  4 83.38 0.5519
26  4 83.79 0.5029
27  4 84.13 0.4582
28  4 84.41 0.4175
29  4 84.64 0.3804
30  4 84.84 0.3466
31  4 85.00 0.3158
32  4 85.13 0.2878
33  4 85.24 0.2622
34  4 85.33 0.2389
35  4 85.41 0.2177
36  4 85.47 0.1983
37  4 85.53 0.1807
38  4 85.57 0.1647
39  4 85.61 0.1500
40  5 85.65 0.1367
41  5 85.69 0.1246
42  5 85.73 0.1135
43  5 85.75 0.1034
44  5 85.78 0.0942
45  5 85.80 0.0859
46  5 85.82 0.0782
47  5 85.83 0.0713
48  6 85.94 0.0649
49  6 86.05 0.0592
50  6 86.14 0.0539
51  6 86.21 0.0491
52  6 86.27 0.0448
53  6 86.32 0.0408
54  6 86.37 0.0372
55  6 86.40 0.0339
56  6 86.43 0.0309
57  7 86.46 0.0281
58  7 86.49 0.0256
59  7 86.52 0.0233
60  7 86.54 0.0213
61  7 86.56 0.0194
62  7 86.57 0.0177
63  7 86.58 0.0161
64  7 86.59 0.0147
65  7 86.60 0.0134
66  7 86.61 0.0122
67  7 86.61 0.0111
68  7 86.62 0.0101
69  7 86.62 0.0092
70  7 86.63 0.0084
71  7 86.63 0.0076
72  7 86.63 0.0070
73  7 86.63 0.0063
74  7 86.63 0.0058
75  7 86.64 0.0053
76  7 86.64 0.0048
77  7 86.64 0.0044
78  7 86.64 0.0040
\end{verbatim}

\begin{center}\rule{0.5\linewidth}{0.5pt}\end{center}

\subsection{Curvas de coeficientes}\label{curvas-de-coeficientes}

\begin{Shaded}
\begin{Highlighting}[]
\FunctionTok{par}\NormalTok{(}\AttributeTok{mfrow=}\FunctionTok{c}\NormalTok{(}\DecValTok{1}\NormalTok{,}\DecValTok{2}\NormalTok{))}
\FunctionTok{plot}\NormalTok{(fit\_ridge, }\AttributeTok{xvar =} \StringTok{"lambda"}\NormalTok{, }\AttributeTok{label =} \ConstantTok{TRUE}\NormalTok{, }\AttributeTok{main =} \StringTok{"Ridge: caminhos dos coef."}\NormalTok{)}
\FunctionTok{plot}\NormalTok{(fit\_lasso, }\AttributeTok{xvar =} \StringTok{"lambda"}\NormalTok{, }\AttributeTok{label =} \ConstantTok{TRUE}\NormalTok{, }\AttributeTok{main =} \StringTok{"Lasso: caminhos dos coef."}\NormalTok{)}
\end{Highlighting}
\end{Shaded}

\pandocbounded{\includegraphics[keepaspectratio]{Aula6_files/figure-pdf/unnamed-chunk-3-1.pdf}}

\begin{Shaded}
\begin{Highlighting}[]
\FunctionTok{par}\NormalTok{(}\AttributeTok{mfrow=}\FunctionTok{c}\NormalTok{(}\DecValTok{1}\NormalTok{,}\DecValTok{1}\NormalTok{))}
\end{Highlighting}
\end{Shaded}

\begin{quote}
Observe que o \textbf{lasso} reduz coeficientes a zero à medida que
\(\lambda\) cresce; o \textbf{ridge} apenas contrai.
\end{quote}

\begin{center}\rule{0.5\linewidth}{0.5pt}\end{center}

\subsection{\texorpdfstring{Escolha de \(\lambda\) por validação
cruzada}{Escolha de \textbackslash lambda por validação cruzada}}\label{escolha-de-lambda-por-validauxe7uxe3o-cruzada}

\begin{Shaded}
\begin{Highlighting}[]
\FunctionTok{set.seed}\NormalTok{(}\DecValTok{123}\NormalTok{)}
\NormalTok{cv\_ridge }\OtherTok{\textless{}{-}} \FunctionTok{cv.glmnet}\NormalTok{(X, y, }\AttributeTok{alpha =} \DecValTok{0}\NormalTok{, }\AttributeTok{nfolds =} \DecValTok{10}\NormalTok{)}
\NormalTok{cv\_lasso }\OtherTok{\textless{}{-}} \FunctionTok{cv.glmnet}\NormalTok{(X, y, }\AttributeTok{alpha =} \DecValTok{1}\NormalTok{, }\AttributeTok{nfolds =} \DecValTok{10}\NormalTok{)}
\NormalTok{cv\_ridge}\SpecialCharTok{$}\NormalTok{lambda.min; cv\_ridge}\SpecialCharTok{$}\NormalTok{lambda}\FloatTok{.1}\NormalTok{se}
\end{Highlighting}
\end{Shaded}

\begin{verbatim}
[1] 0.5146981
\end{verbatim}

\begin{verbatim}
[1] 4.373665
\end{verbatim}

\begin{Shaded}
\begin{Highlighting}[]
\NormalTok{cv\_lasso}\SpecialCharTok{$}\NormalTok{lambda.min; cv\_lasso}\SpecialCharTok{$}\NormalTok{lambda}\FloatTok{.1}\NormalTok{se}
\end{Highlighting}
\end{Shaded}

\begin{verbatim}
[1] 0.1134977
\end{verbatim}

\begin{verbatim}
[1] 1.274947
\end{verbatim}

\begin{Shaded}
\begin{Highlighting}[]
\FunctionTok{par}\NormalTok{(}\AttributeTok{mfrow=}\FunctionTok{c}\NormalTok{(}\DecValTok{1}\NormalTok{,}\DecValTok{2}\NormalTok{))}
\FunctionTok{plot}\NormalTok{(cv\_ridge); }\FunctionTok{title}\NormalTok{(}\StringTok{"CV Ridge"}\NormalTok{, }\AttributeTok{line =} \FloatTok{2.5}\NormalTok{)}
\FunctionTok{plot}\NormalTok{(cv\_lasso); }\FunctionTok{title}\NormalTok{(}\StringTok{"CV Lasso"}\NormalTok{, }\AttributeTok{line =} \FloatTok{2.5}\NormalTok{)}
\end{Highlighting}
\end{Shaded}

\pandocbounded{\includegraphics[keepaspectratio]{Aula6_files/figure-pdf/unnamed-chunk-4-1.pdf}}

\begin{Shaded}
\begin{Highlighting}[]
\FunctionTok{par}\NormalTok{(}\AttributeTok{mfrow=}\FunctionTok{c}\NormalTok{(}\DecValTok{1}\NormalTok{,}\DecValTok{1}\NormalTok{))}
\end{Highlighting}
\end{Shaded}

\begin{center}\rule{0.5\linewidth}{0.5pt}\end{center}

\subsection{Modelos finais e
predição}\label{modelos-finais-e-prediuxe7uxe3o}

\begin{Shaded}
\begin{Highlighting}[]
\FunctionTok{set.seed}\NormalTok{(}\DecValTok{1}\NormalTok{)}
\NormalTok{idx }\OtherTok{\textless{}{-}} \FunctionTok{sample}\NormalTok{(}\FunctionTok{seq\_len}\NormalTok{(}\FunctionTok{nrow}\NormalTok{(X)), }\AttributeTok{size =} \FunctionTok{round}\NormalTok{(}\FloatTok{0.7}\SpecialCharTok{*}\FunctionTok{nrow}\NormalTok{(X)))}
\NormalTok{Xtr }\OtherTok{\textless{}{-}}\NormalTok{ X[idx,]; ytr }\OtherTok{\textless{}{-}}\NormalTok{ y[idx]}
\NormalTok{Xte }\OtherTok{\textless{}{-}}\NormalTok{ X[}\SpecialCharTok{{-}}\NormalTok{idx,]; yte }\OtherTok{\textless{}{-}}\NormalTok{ y[}\SpecialCharTok{{-}}\NormalTok{idx]}
\NormalTok{cv\_ridge2 }\OtherTok{\textless{}{-}} \FunctionTok{cv.glmnet}\NormalTok{(Xtr, ytr, }\AttributeTok{alpha =} \DecValTok{0}\NormalTok{)}
\NormalTok{cv\_lasso2 }\OtherTok{\textless{}{-}} \FunctionTok{cv.glmnet}\NormalTok{(Xtr, ytr, }\AttributeTok{alpha =} \DecValTok{1}\NormalTok{)}
\NormalTok{pr\_ridge }\OtherTok{\textless{}{-}} \FunctionTok{predict}\NormalTok{(cv\_ridge2, }\AttributeTok{newx =}\NormalTok{ Xte, }\AttributeTok{s =} \StringTok{"lambda.min"}\NormalTok{)}
\NormalTok{pr\_lasso }\OtherTok{\textless{}{-}} \FunctionTok{predict}\NormalTok{(cv\_lasso2, }\AttributeTok{newx =}\NormalTok{ Xte, }\AttributeTok{s =} \StringTok{"lambda.min"}\NormalTok{)}
\NormalTok{rmse }\OtherTok{\textless{}{-}} \ControlFlowTok{function}\NormalTok{(a,b) }\FunctionTok{sqrt}\NormalTok{(}\FunctionTok{mean}\NormalTok{((a}\SpecialCharTok{{-}}\NormalTok{b)}\SpecialCharTok{\^{}}\DecValTok{2}\NormalTok{))}
\NormalTok{res }\OtherTok{\textless{}{-}} \FunctionTok{tibble}\NormalTok{(}
  \AttributeTok{modelo =} \FunctionTok{c}\NormalTok{(}\StringTok{"ridge"}\NormalTok{,}\StringTok{"lasso"}\NormalTok{),}
  \AttributeTok{RMSE =} \FunctionTok{c}\NormalTok{(}\FunctionTok{rmse}\NormalTok{(yte, pr\_ridge[,}\DecValTok{1}\NormalTok{]), }\FunctionTok{rmse}\NormalTok{(yte, pr\_lasso[,}\DecValTok{1}\NormalTok{]))}
\NormalTok{)}
\NormalTok{res}
\end{Highlighting}
\end{Shaded}

\begin{verbatim}
# A tibble: 2 x 2
  modelo  RMSE
  <chr>  <dbl>
1 ridge   2.73
2 lasso   2.63
\end{verbatim}

\section{Exemplo 2 (p \textgreater{} n / correlações): dados
sintéticos}\label{exemplo-2-p-n-correlauxe7uxf5es-dados-sintuxe9ticos}

Gerar dados com colinearidade forte e \(p=20\), \(n=60\).

\begin{Shaded}
\begin{Highlighting}[]
\FunctionTok{set.seed}\NormalTok{(}\DecValTok{321}\NormalTok{)}
\NormalTok{n }\OtherTok{\textless{}{-}} \DecValTok{60}\NormalTok{; p }\OtherTok{\textless{}{-}} \DecValTok{20}
\NormalTok{Sigma }\OtherTok{\textless{}{-}} \FloatTok{0.7}\SpecialCharTok{\^{}}\FunctionTok{abs}\NormalTok{(}\FunctionTok{outer}\NormalTok{(}\DecValTok{1}\SpecialCharTok{:}\NormalTok{p, }\DecValTok{1}\SpecialCharTok{:}\NormalTok{p, }\StringTok{"{-}"}\NormalTok{))  }\CommentTok{\# matriz Toeplitz (AR1)}
\NormalTok{Z }\OtherTok{\textless{}{-}}\NormalTok{ MASS}\SpecialCharTok{::}\FunctionTok{mvrnorm}\NormalTok{(n, }\AttributeTok{mu =} \FunctionTok{rep}\NormalTok{(}\DecValTok{0}\NormalTok{, p), }\AttributeTok{Sigma =}\NormalTok{ Sigma)}
\NormalTok{beta\_true }\OtherTok{\textless{}{-}} \FunctionTok{c}\NormalTok{(}\FunctionTok{runif}\NormalTok{(}\DecValTok{5}\NormalTok{, }\DecValTok{2}\NormalTok{, }\DecValTok{4}\NormalTok{), }\FunctionTok{rep}\NormalTok{(}\DecValTok{0}\NormalTok{, p}\DecValTok{{-}5}\NormalTok{))}
\NormalTok{y }\OtherTok{\textless{}{-}}\NormalTok{ Z }\SpecialCharTok{\%*\%}\NormalTok{ beta\_true }\SpecialCharTok{+} \FunctionTok{rnorm}\NormalTok{(n, }\AttributeTok{sd=}\DecValTok{3}\NormalTok{)}
\NormalTok{X }\OtherTok{\textless{}{-}} \FunctionTok{scale}\NormalTok{(Z)}
\NormalTok{cv\_r }\OtherTok{\textless{}{-}} \FunctionTok{cv.glmnet}\NormalTok{(X, y, }\AttributeTok{alpha =} \DecValTok{0}\NormalTok{)}
\NormalTok{cv\_l }\OtherTok{\textless{}{-}} \FunctionTok{cv.glmnet}\NormalTok{(X, y, }\AttributeTok{alpha =} \DecValTok{1}\NormalTok{)}
\FunctionTok{coef}\NormalTok{(cv\_l, }\AttributeTok{s =} \StringTok{"lambda.min"}\NormalTok{) }\SpecialCharTok{|\textgreater{}} \FunctionTok{as.matrix}\NormalTok{() }\SpecialCharTok{|\textgreater{}} \FunctionTok{head}\NormalTok{(}\DecValTok{25}\NormalTok{)}
\end{Highlighting}
\end{Shaded}

\begin{verbatim}
            lambda.min
(Intercept)  0.1993408
V1           2.7608709
V2           1.8651526
V3           4.6950655
V4           2.6950877
V5           2.5974331
V6           0.0000000
V7           0.0000000
V8           0.0000000
V9           0.0000000
V10          0.0000000
V11          0.0000000
V12          0.0000000
V13          0.0000000
V14          0.0000000
V15          0.0000000
V16          0.0000000
V17          0.0000000
V18          0.0000000
V19          0.6416292
V20          0.1389642
\end{verbatim}

\begin{itemize}
\tightlist
\item
  \textbf{Lasso} recupera esparsidade (muitos coef. exatos 0).
\item
  \textbf{Ridge} tende a reter todos os preditores com pesos pequenos.
\end{itemize}

\section{Comparação teórica
resumida}\label{comparauxe7uxe3o-teuxf3rica-resumida}

\begin{longtable}[]{@{}lll@{}}
\toprule\noalign{}
Propriedade & Ridge (L2) & Lasso (L1) \\
\midrule\noalign{}
\endhead
\bottomrule\noalign{}
\endlastfoot
Contração & Contínua & Contínua + zeros \\
Seleção de variáveis & Não & Sim \\
Colinearidade forte & Distribui pesos & Escolhe um (instável) \\
p \textgreater{} n & Funciona & Funciona \\
Solução fechada & Sim & Não \\
Viés & Maior & Pode ser maior \\
Variância & Menor & Menor (tipicamente) \\
\end{longtable}

\begin{quote}
\textbf{Regra de bolso:} quando há \textbf{muitos preditores
correlacionados}, \textbf{Elastic Net} (combina L1+L2) é frequentemente
superior.
\end{quote}

\section{\texorpdfstring{Escolha de \(\lambda\) e
métricas}{Escolha de \textbackslash lambda e métricas}}\label{escolha-de-lambda-e-muxe9tricas}

\begin{itemize}
\tightlist
\item
  \textbf{CV k-fold} (padrão em \texttt{cv.glmnet}): escolher
  \texttt{lambda.min} (menor erro) ou \texttt{lambda.1se} (mais
  parcimonioso).
\item
  Métricas: RMSE, MAE, \(R^2\) em validação/teste.
\item
  Reportar \textbf{curvas de CV} e \textbf{caminhos de coeficientes}.
\end{itemize}

\section{Limitações e extensões}\label{limitauxe7uxf5es-e-extensuxf5es}

\begin{itemize}
\tightlist
\item
  Lasso pode ser instável com preditores muito correlacionados.
\item
  Ridge não faz seleção automática.
\item
  \textbf{Elastic Net}: \(\alpha\in[0,1]\) interpola Lasso (1) e Ridge
  (0).
\end{itemize}

\begin{Shaded}
\begin{Highlighting}[]
\CommentTok{\# Exemplo rápido: elastic net com alpha=0.5}
\NormalTok{cv\_en }\OtherTok{\textless{}{-}} \FunctionTok{cv.glmnet}\NormalTok{(X, y, }\AttributeTok{alpha =} \FloatTok{0.5}\NormalTok{)}
\FunctionTok{coef}\NormalTok{(cv\_en, }\AttributeTok{s =} \StringTok{"lambda.1se"}\NormalTok{) }\SpecialCharTok{|\textgreater{}} \FunctionTok{as.matrix}\NormalTok{() }\SpecialCharTok{|\textgreater{}} \FunctionTok{head}\NormalTok{()}
\end{Highlighting}
\end{Shaded}

\begin{verbatim}
            lambda.1se
(Intercept)  0.1993408
V1           2.3234317
V2           2.1301774
V3           4.0862506
V4           2.7969578
V5           2.1435526
\end{verbatim}

\section{Predição e incerteza}\label{prediuxe7uxe3o-e-incerteza}

\begin{itemize}
\tightlist
\item
  Predição pontual via \texttt{predict}.
\item
  Intervalos padrão de MQO não se aplicam diretamente; \texttt{glmnet}
  foca predição.
\item
  Para \textbf{intervalos de previsão}: usar re-amostragem (bootstrap)
  ou métodos pós-seleção.
\end{itemize}

\begin{Shaded}
\begin{Highlighting}[]
\FunctionTok{set.seed}\NormalTok{(}\DecValTok{99}\NormalTok{)}
\CommentTok{\# Bootstrap simples do modelo lasso (ilustrativo)}
\NormalTok{B }\OtherTok{\textless{}{-}} \DecValTok{200}
\NormalTok{preds }\OtherTok{\textless{}{-}} \FunctionTok{replicate}\NormalTok{(B, \{}
\NormalTok{  id }\OtherTok{\textless{}{-}} \FunctionTok{sample}\NormalTok{(}\FunctionTok{seq\_len}\NormalTok{(}\FunctionTok{nrow}\NormalTok{(X)), }\AttributeTok{replace=}\ConstantTok{TRUE}\NormalTok{)}
\NormalTok{  fit }\OtherTok{\textless{}{-}} \FunctionTok{cv.glmnet}\NormalTok{(X[id,], y[id], }\AttributeTok{alpha=}\DecValTok{1}\NormalTok{)}
  \FunctionTok{as.numeric}\NormalTok{(}\FunctionTok{predict}\NormalTok{(fit, }\AttributeTok{newx =}\NormalTok{ X[}\DecValTok{1}\NormalTok{,,}\AttributeTok{drop=}\ConstantTok{FALSE}\NormalTok{], }\AttributeTok{s=}\StringTok{"lambda.min"}\NormalTok{))}
\NormalTok{\})}
\FunctionTok{quantile}\NormalTok{(preds, }\FunctionTok{c}\NormalTok{(.}\DecValTok{025}\NormalTok{,.}\DecValTok{5}\NormalTok{,.}\DecValTok{975}\NormalTok{))}
\end{Highlighting}
\end{Shaded}

\begin{verbatim}
     2.5%       50%     97.5% 
-28.53620 -25.16784 -21.60128 
\end{verbatim}

\section{Exemplo final}\label{exemplo-final}

\begin{Shaded}
\begin{Highlighting}[]
\FunctionTok{library}\NormalTok{(dplyr)}
\FunctionTok{set.seed}\NormalTok{(}\DecValTok{7}\NormalTok{)}
\NormalTok{idx }\OtherTok{\textless{}{-}} \FunctionTok{sample}\NormalTok{(}\FunctionTok{seq\_len}\NormalTok{(}\FunctionTok{nrow}\NormalTok{(mtcars)), }\AttributeTok{size =} \DecValTok{24}\NormalTok{)}
\NormalTok{train }\OtherTok{\textless{}{-}}\NormalTok{ mtcars[idx,]; test }\OtherTok{\textless{}{-}}\NormalTok{ mtcars[}\SpecialCharTok{{-}}\NormalTok{idx,]}
\NormalTok{Xtr }\OtherTok{\textless{}{-}} \FunctionTok{model.matrix}\NormalTok{(mpg }\SpecialCharTok{\textasciitilde{}}\NormalTok{ ., train)[,}\SpecialCharTok{{-}}\DecValTok{1}\NormalTok{]; ytr }\OtherTok{\textless{}{-}}\NormalTok{ train}\SpecialCharTok{$}\NormalTok{mpg}
\NormalTok{Xte }\OtherTok{\textless{}{-}} \FunctionTok{model.matrix}\NormalTok{(mpg }\SpecialCharTok{\textasciitilde{}}\NormalTok{ ., test)[,}\SpecialCharTok{{-}}\DecValTok{1}\NormalTok{];  yte }\OtherTok{\textless{}{-}}\NormalTok{ test}\SpecialCharTok{$}\NormalTok{mpg}
\NormalTok{cv\_las }\OtherTok{\textless{}{-}} \FunctionTok{cv.glmnet}\NormalTok{(Xtr, ytr, }\AttributeTok{alpha=}\DecValTok{1}\NormalTok{)}
\NormalTok{pr }\OtherTok{\textless{}{-}} \FunctionTok{predict}\NormalTok{(cv\_las, }\AttributeTok{newx =}\NormalTok{ Xte, }\AttributeTok{s =} \StringTok{"lambda.1se"}\NormalTok{)}
\NormalTok{rmse }\OtherTok{\textless{}{-}} \FunctionTok{sqrt}\NormalTok{(}\FunctionTok{mean}\NormalTok{((yte }\SpecialCharTok{{-}}\NormalTok{ pr[,}\DecValTok{1}\NormalTok{])}\SpecialCharTok{\^{}}\DecValTok{2}\NormalTok{))}
\FunctionTok{list}\NormalTok{(}\AttributeTok{lambda\_1se =}\NormalTok{ cv\_las}\SpecialCharTok{$}\NormalTok{lambda}\FloatTok{.1}\NormalTok{se, }\AttributeTok{RMSE\_teste =}\NormalTok{ rmse)}
\end{Highlighting}
\end{Shaded}

\begin{verbatim}
$lambda_1se
[1] 1.794338

$RMSE_teste
[1] 3.198099
\end{verbatim}

\section{Exercícios}\label{exercuxedcios}

\begin{enumerate}
\def\labelenumi{\arabic{enumi}.}
\tightlist
\item
  Reproduza o Exemplo 1 com outro \emph{target} (e.g., \texttt{wt}) e
  documente diferenças.
\end{enumerate}

\begin{Shaded}
\begin{Highlighting}[]
\FunctionTok{set.seed}\NormalTok{(}\DecValTok{123}\NormalTok{)}
\NormalTok{D }\OtherTok{\textless{}{-}}\NormalTok{ mtcars }\SpecialCharTok{|\textgreater{}} \FunctionTok{mutate}\NormalTok{(}\FunctionTok{across}\NormalTok{(}\FunctionTok{c}\NormalTok{(am, gear), factor))}
\CommentTok{\# Matriz modelo (one{-}hot sem intercepto)}
\NormalTok{X }\OtherTok{\textless{}{-}} \FunctionTok{model.matrix}\NormalTok{(wt }\SpecialCharTok{\textasciitilde{}}\NormalTok{ disp }\SpecialCharTok{+}\NormalTok{ hp }\SpecialCharTok{+}\NormalTok{ wt }\SpecialCharTok{+}\NormalTok{ qsec }\SpecialCharTok{+}\NormalTok{ am }\SpecialCharTok{+}\NormalTok{ gear, }\AttributeTok{data =}\NormalTok{ D)[, }\SpecialCharTok{{-}}\DecValTok{1}\NormalTok{]}
\NormalTok{y }\OtherTok{\textless{}{-}}\NormalTok{ D}\SpecialCharTok{$}\NormalTok{wt}
\CommentTok{\# grid de lambdas padrão do glmnet}
\NormalTok{fit\_ridge }\OtherTok{\textless{}{-}} \FunctionTok{glmnet}\NormalTok{(X, y, }\AttributeTok{alpha =} \DecValTok{0}\NormalTok{)   }\CommentTok{\# ridge}
\NormalTok{fit\_lasso }\OtherTok{\textless{}{-}} \FunctionTok{glmnet}\NormalTok{(X, y, }\AttributeTok{alpha =} \DecValTok{1}\NormalTok{)   }\CommentTok{\# lasso}
\NormalTok{fit\_ridge; fit\_lasso}
\FunctionTok{par}\NormalTok{(}\AttributeTok{mfrow=}\FunctionTok{c}\NormalTok{(}\DecValTok{1}\NormalTok{,}\DecValTok{2}\NormalTok{))}
\FunctionTok{plot}\NormalTok{(fit\_ridge, }\AttributeTok{xvar =} \StringTok{"lambda"}\NormalTok{, }\AttributeTok{label =} \ConstantTok{TRUE}\NormalTok{, }\AttributeTok{main =} \StringTok{"Ridge: caminhos dos coef."}\NormalTok{)}
\FunctionTok{plot}\NormalTok{(fit\_lasso, }\AttributeTok{xvar =} \StringTok{"lambda"}\NormalTok{, }\AttributeTok{label =} \ConstantTok{TRUE}\NormalTok{, }\AttributeTok{main =} \StringTok{"Lasso: caminhos dos coef."}\NormalTok{)}
\FunctionTok{par}\NormalTok{(}\AttributeTok{mfrow=}\FunctionTok{c}\NormalTok{(}\DecValTok{1}\NormalTok{,}\DecValTok{1}\NormalTok{))}
\end{Highlighting}
\end{Shaded}

\begin{enumerate}
\def\labelenumi{\arabic{enumi}.}
\setcounter{enumi}{1}
\tightlist
\item
  Em dados sintéticos (p\textgreater n), compare estabilidade da seleção
  do Lasso com 100 \emph{splits}.
\item
  Use \textbf{Elastic Net} e faça \emph{grid search} em
  \(\alpha \in (0, 0.25, 0.5, 0.75, 1)\), reportando RMSE de teste e nº
  de variáveis selecionadas.
\item
  Represente graficamente as \textbf{curvas dos coeficientes} e
  interprete.
\end{enumerate}

\section{Referências}\label{referuxeancias}

\begin{itemize}
\tightlist
\item
  Hastie, Tibshirani \& Friedman (2009). \emph{The Elements of
  Statistical Learning}.
\item
  James et al.~(2021). \emph{An Introduction to Statistical Learning},
  cap. 6.
\item
  Documentação \texttt{glmnet} (vignette): \texttt{vignette("glmnet")}.
\end{itemize}

\begin{center}\rule{0.5\linewidth}{0.5pt}\end{center}




\end{document}
