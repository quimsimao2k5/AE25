% Options for packages loaded elsewhere
% Options for packages loaded elsewhere
\PassOptionsToPackage{unicode}{hyperref}
\PassOptionsToPackage{hyphens}{url}
\PassOptionsToPackage{dvipsnames,svgnames,x11names}{xcolor}
%
\documentclass[
  portuguese,
  letterpaper,
  DIV=11,
  numbers=noendperiod]{scrartcl}
\usepackage{xcolor}
\usepackage{amsmath,amssymb}
\setcounter{secnumdepth}{-\maxdimen} % remove section numbering
\usepackage{iftex}
\ifPDFTeX
  \usepackage[T1]{fontenc}
  \usepackage[utf8]{inputenc}
  \usepackage{textcomp} % provide euro and other symbols
\else % if luatex or xetex
  \usepackage{unicode-math} % this also loads fontspec
  \defaultfontfeatures{Scale=MatchLowercase}
  \defaultfontfeatures[\rmfamily]{Ligatures=TeX,Scale=1}
\fi
\usepackage{lmodern}
\ifPDFTeX\else
  % xetex/luatex font selection
\fi
% Use upquote if available, for straight quotes in verbatim environments
\IfFileExists{upquote.sty}{\usepackage{upquote}}{}
\IfFileExists{microtype.sty}{% use microtype if available
  \usepackage[]{microtype}
  \UseMicrotypeSet[protrusion]{basicmath} % disable protrusion for tt fonts
}{}
\makeatletter
\@ifundefined{KOMAClassName}{% if non-KOMA class
  \IfFileExists{parskip.sty}{%
    \usepackage{parskip}
  }{% else
    \setlength{\parindent}{0pt}
    \setlength{\parskip}{6pt plus 2pt minus 1pt}}
}{% if KOMA class
  \KOMAoptions{parskip=half}}
\makeatother
% Make \paragraph and \subparagraph free-standing
\makeatletter
\ifx\paragraph\undefined\else
  \let\oldparagraph\paragraph
  \renewcommand{\paragraph}{
    \@ifstar
      \xxxParagraphStar
      \xxxParagraphNoStar
  }
  \newcommand{\xxxParagraphStar}[1]{\oldparagraph*{#1}\mbox{}}
  \newcommand{\xxxParagraphNoStar}[1]{\oldparagraph{#1}\mbox{}}
\fi
\ifx\subparagraph\undefined\else
  \let\oldsubparagraph\subparagraph
  \renewcommand{\subparagraph}{
    \@ifstar
      \xxxSubParagraphStar
      \xxxSubParagraphNoStar
  }
  \newcommand{\xxxSubParagraphStar}[1]{\oldsubparagraph*{#1}\mbox{}}
  \newcommand{\xxxSubParagraphNoStar}[1]{\oldsubparagraph{#1}\mbox{}}
\fi
\makeatother

\usepackage{color}
\usepackage{fancyvrb}
\newcommand{\VerbBar}{|}
\newcommand{\VERB}{\Verb[commandchars=\\\{\}]}
\DefineVerbatimEnvironment{Highlighting}{Verbatim}{commandchars=\\\{\}}
% Add ',fontsize=\small' for more characters per line
\usepackage{framed}
\definecolor{shadecolor}{RGB}{241,243,245}
\newenvironment{Shaded}{\begin{snugshade}}{\end{snugshade}}
\newcommand{\AlertTok}[1]{\textcolor[rgb]{0.68,0.00,0.00}{#1}}
\newcommand{\AnnotationTok}[1]{\textcolor[rgb]{0.37,0.37,0.37}{#1}}
\newcommand{\AttributeTok}[1]{\textcolor[rgb]{0.40,0.45,0.13}{#1}}
\newcommand{\BaseNTok}[1]{\textcolor[rgb]{0.68,0.00,0.00}{#1}}
\newcommand{\BuiltInTok}[1]{\textcolor[rgb]{0.00,0.23,0.31}{#1}}
\newcommand{\CharTok}[1]{\textcolor[rgb]{0.13,0.47,0.30}{#1}}
\newcommand{\CommentTok}[1]{\textcolor[rgb]{0.37,0.37,0.37}{#1}}
\newcommand{\CommentVarTok}[1]{\textcolor[rgb]{0.37,0.37,0.37}{\textit{#1}}}
\newcommand{\ConstantTok}[1]{\textcolor[rgb]{0.56,0.35,0.01}{#1}}
\newcommand{\ControlFlowTok}[1]{\textcolor[rgb]{0.00,0.23,0.31}{\textbf{#1}}}
\newcommand{\DataTypeTok}[1]{\textcolor[rgb]{0.68,0.00,0.00}{#1}}
\newcommand{\DecValTok}[1]{\textcolor[rgb]{0.68,0.00,0.00}{#1}}
\newcommand{\DocumentationTok}[1]{\textcolor[rgb]{0.37,0.37,0.37}{\textit{#1}}}
\newcommand{\ErrorTok}[1]{\textcolor[rgb]{0.68,0.00,0.00}{#1}}
\newcommand{\ExtensionTok}[1]{\textcolor[rgb]{0.00,0.23,0.31}{#1}}
\newcommand{\FloatTok}[1]{\textcolor[rgb]{0.68,0.00,0.00}{#1}}
\newcommand{\FunctionTok}[1]{\textcolor[rgb]{0.28,0.35,0.67}{#1}}
\newcommand{\ImportTok}[1]{\textcolor[rgb]{0.00,0.46,0.62}{#1}}
\newcommand{\InformationTok}[1]{\textcolor[rgb]{0.37,0.37,0.37}{#1}}
\newcommand{\KeywordTok}[1]{\textcolor[rgb]{0.00,0.23,0.31}{\textbf{#1}}}
\newcommand{\NormalTok}[1]{\textcolor[rgb]{0.00,0.23,0.31}{#1}}
\newcommand{\OperatorTok}[1]{\textcolor[rgb]{0.37,0.37,0.37}{#1}}
\newcommand{\OtherTok}[1]{\textcolor[rgb]{0.00,0.23,0.31}{#1}}
\newcommand{\PreprocessorTok}[1]{\textcolor[rgb]{0.68,0.00,0.00}{#1}}
\newcommand{\RegionMarkerTok}[1]{\textcolor[rgb]{0.00,0.23,0.31}{#1}}
\newcommand{\SpecialCharTok}[1]{\textcolor[rgb]{0.37,0.37,0.37}{#1}}
\newcommand{\SpecialStringTok}[1]{\textcolor[rgb]{0.13,0.47,0.30}{#1}}
\newcommand{\StringTok}[1]{\textcolor[rgb]{0.13,0.47,0.30}{#1}}
\newcommand{\VariableTok}[1]{\textcolor[rgb]{0.07,0.07,0.07}{#1}}
\newcommand{\VerbatimStringTok}[1]{\textcolor[rgb]{0.13,0.47,0.30}{#1}}
\newcommand{\WarningTok}[1]{\textcolor[rgb]{0.37,0.37,0.37}{\textit{#1}}}

\usepackage{longtable,booktabs,array}
\usepackage{calc} % for calculating minipage widths
% Correct order of tables after \paragraph or \subparagraph
\usepackage{etoolbox}
\makeatletter
\patchcmd\longtable{\par}{\if@noskipsec\mbox{}\fi\par}{}{}
\makeatother
% Allow footnotes in longtable head/foot
\IfFileExists{footnotehyper.sty}{\usepackage{footnotehyper}}{\usepackage{footnote}}
\makesavenoteenv{longtable}
\usepackage{graphicx}
\makeatletter
\newsavebox\pandoc@box
\newcommand*\pandocbounded[1]{% scales image to fit in text height/width
  \sbox\pandoc@box{#1}%
  \Gscale@div\@tempa{\textheight}{\dimexpr\ht\pandoc@box+\dp\pandoc@box\relax}%
  \Gscale@div\@tempb{\linewidth}{\wd\pandoc@box}%
  \ifdim\@tempb\p@<\@tempa\p@\let\@tempa\@tempb\fi% select the smaller of both
  \ifdim\@tempa\p@<\p@\scalebox{\@tempa}{\usebox\pandoc@box}%
  \else\usebox{\pandoc@box}%
  \fi%
}
% Set default figure placement to htbp
\def\fps@figure{htbp}
\makeatother



\ifLuaTeX
\usepackage[bidi=basic]{babel}
\else
\usepackage[bidi=default]{babel}
\fi
% get rid of language-specific shorthands (see #6817):
\let\LanguageShortHands\languageshorthands
\def\languageshorthands#1{}


\setlength{\emergencystretch}{3em} % prevent overfull lines

\providecommand{\tightlist}{%
  \setlength{\itemsep}{0pt}\setlength{\parskip}{0pt}}



 


\KOMAoption{captions}{tableheading}
\makeatletter
\@ifpackageloaded{caption}{}{\usepackage{caption}}
\AtBeginDocument{%
\ifdefined\contentsname
  \renewcommand*\contentsname{Índice}
\else
  \newcommand\contentsname{Índice}
\fi
\ifdefined\listfigurename
  \renewcommand*\listfigurename{Lista de Figuras}
\else
  \newcommand\listfigurename{Lista de Figuras}
\fi
\ifdefined\listtablename
  \renewcommand*\listtablename{Lista de Tabelas}
\else
  \newcommand\listtablename{Lista de Tabelas}
\fi
\ifdefined\figurename
  \renewcommand*\figurename{Figura}
\else
  \newcommand\figurename{Figura}
\fi
\ifdefined\tablename
  \renewcommand*\tablename{Tabela}
\else
  \newcommand\tablename{Tabela}
\fi
}
\@ifpackageloaded{float}{}{\usepackage{float}}
\floatstyle{ruled}
\@ifundefined{c@chapter}{\newfloat{codelisting}{h}{lop}}{\newfloat{codelisting}{h}{lop}[chapter]}
\floatname{codelisting}{Listagem}
\newcommand*\listoflistings{\listof{codelisting}{Lista de Listagens}}
\makeatother
\makeatletter
\makeatother
\makeatletter
\@ifpackageloaded{caption}{}{\usepackage{caption}}
\@ifpackageloaded{subcaption}{}{\usepackage{subcaption}}
\makeatother
\usepackage{bookmark}
\IfFileExists{xurl.sty}{\usepackage{xurl}}{} % add URL line breaks if available
\urlstyle{same}
\hypersetup{
  pdftitle={Regressão Linear},
  pdfauthor={Soraia Pereira},
  pdflang={pt},
  colorlinks=true,
  linkcolor={blue},
  filecolor={Maroon},
  citecolor={Blue},
  urlcolor={Blue},
  pdfcreator={LaTeX via pandoc}}


\title{Regressão Linear}
\author{Soraia Pereira}
\date{2025-11-05}
\begin{document}
\maketitle


\section{AULA 1 Regressão Linear Simples
(SLR)}\label{aula-1-regressuxe3o-linear-simples-slr}

\subsection{Motivação: prever e
explicar}\label{motivauxe7uxe3o-prever-e-explicar}

\begin{itemize}
\tightlist
\item
  Objetivo: modelar \(\mathbb{E}[Y\mid X]\) (reta média condicional) e
  \textbf{prever} \(Y\) para novos \(X\).
\item
  Dois usos:

  \begin{itemize}
  \tightlist
  \item
    \textbf{Explicação}: interpretar \(\beta_1\) (efeito marginal de
    \(X\)).
  \item
    \textbf{Previsão}: obter \(\widehat{Y}\) com incerteza (IC/IP).
  \end{itemize}
\end{itemize}

\begin{Shaded}
\begin{Highlighting}[]
\CommentTok{\# Exemplo base que iremos usar}
\FunctionTok{data}\NormalTok{(mtcars)}
\FunctionTok{head}\NormalTok{(mtcars[, }\FunctionTok{c}\NormalTok{(}\StringTok{"mpg"}\NormalTok{, }\StringTok{"wt"}\NormalTok{)], }\DecValTok{5}\NormalTok{)}
\end{Highlighting}
\end{Shaded}

\begin{verbatim}
                   mpg    wt
Mazda RX4         21.0 2.620
Mazda RX4 Wag     21.0 2.875
Datsun 710        22.8 2.320
Hornet 4 Drive    21.4 3.215
Hornet Sportabout 18.7 3.440
\end{verbatim}

\begin{center}\rule{0.5\linewidth}{0.5pt}\end{center}

\subsection{Modelo SLR}\label{modelo-slr}

\[
Y_i=\beta_0+\beta_1 X_i+\varepsilon_i,\quad i=1,\dots,n
\]

\begin{itemize}
\tightlist
\item
  \(\beta_0\) (intercept), \(\beta_1\) (declive).
\item
  \(\varepsilon_i\) capta variabilidade não explicada;
  \(\mathbb{E}[\varepsilon_i]=0\);
  \(\mathrm{Var}(\varepsilon_i)=\sigma^2\).
\end{itemize}

\begin{Shaded}
\begin{Highlighting}[]
\CommentTok{\# Ajuste do modelo SLR: mpg \textasciitilde{} wt}
\NormalTok{m\_slr }\OtherTok{\textless{}{-}} \FunctionTok{lm}\NormalTok{(mpg }\SpecialCharTok{\textasciitilde{}}\NormalTok{ wt, }\AttributeTok{data =}\NormalTok{ mtcars)}
\FunctionTok{coef}\NormalTok{(m\_slr)}
\end{Highlighting}
\end{Shaded}

\begin{verbatim}
(Intercept)          wt 
  37.285126   -5.344472 
\end{verbatim}

\begin{center}\rule{0.5\linewidth}{0.5pt}\end{center}

\subsection{Linearidade na média
(conceito)}\label{linearidade-na-muxe9dia-conceito}

\[
\mathbb{E}[Y\mid X] = \beta_0+\beta_1 X
\]

\begin{itemize}
\tightlist
\item
  Linearidade é sobre a \textbf{média condicional}, não sobre cada
  observação.
\end{itemize}

\begin{Shaded}
\begin{Highlighting}[]
\CommentTok{\# Gráfico de dispersão + reta ajustada}
\FunctionTok{plot}\NormalTok{(mtcars}\SpecialCharTok{$}\NormalTok{wt, mtcars}\SpecialCharTok{$}\NormalTok{mpg,}
     \AttributeTok{pch=}\DecValTok{19}\NormalTok{, }\AttributeTok{xlab=}\StringTok{"Peso (1000 lbs)"}\NormalTok{, }\AttributeTok{ylab=}\StringTok{"Consumo (mpg)"}\NormalTok{)}
\FunctionTok{abline}\NormalTok{(m\_slr, }\AttributeTok{col=}\StringTok{"tomato"}\NormalTok{, }\AttributeTok{lwd=}\DecValTok{2}\NormalTok{)}
\end{Highlighting}
\end{Shaded}

\pandocbounded{\includegraphics[keepaspectratio]{Aula4_files/figure-pdf/unnamed-chunk-3-1.pdf}}

\begin{center}\rule{0.5\linewidth}{0.5pt}\end{center}

\subsection{Estimação por MMQ (OLS):
ideia}\label{estimauxe7uxe3o-por-mmq-ols-ideia}

Minimizar: \[
\mathrm{RSS}=\sum_{i=1}^n \big(Y_i-\beta_0-\beta_1 X_i\big)^2
\]

\begin{Shaded}
\begin{Highlighting}[]
\CommentTok{\# RSS do modelo}
\NormalTok{rss }\OtherTok{\textless{}{-}} \FunctionTok{sum}\NormalTok{(}\FunctionTok{residuals}\NormalTok{(m\_slr)}\SpecialCharTok{\^{}}\DecValTok{2}\NormalTok{)}
\NormalTok{n }\OtherTok{\textless{}{-}} \FunctionTok{nrow}\NormalTok{(mtcars)}
\NormalTok{rss}
\end{Highlighting}
\end{Shaded}

\begin{verbatim}
[1] 278.3219
\end{verbatim}

\begin{center}\rule{0.5\linewidth}{0.5pt}\end{center}

\subsection{Estimação por MMQ (OLS): expressão
fechada}\label{estimauxe7uxe3o-por-mmq-ols-expressuxe3o-fechada}

\[
\hat\beta_1=\frac{\sum (X_i-\bar X)(Y_i-\bar Y)}{\sum (X_i-\bar X)^2},\qquad
\hat\beta_0=\bar Y-\hat\beta_1 \bar X
\]

\begin{Shaded}
\begin{Highlighting}[]
\CommentTok{\# Cálculo manual de beta1 e beta0}
\NormalTok{x }\OtherTok{\textless{}{-}}\NormalTok{ mtcars}\SpecialCharTok{$}\NormalTok{wt; y }\OtherTok{\textless{}{-}}\NormalTok{ mtcars}\SpecialCharTok{$}\NormalTok{mpg}
\NormalTok{b1 }\OtherTok{\textless{}{-}} \FunctionTok{sum}\NormalTok{( (x}\SpecialCharTok{{-}}\FunctionTok{mean}\NormalTok{(x))}\SpecialCharTok{*}\NormalTok{(y}\SpecialCharTok{{-}}\FunctionTok{mean}\NormalTok{(y)) ) }\SpecialCharTok{/} \FunctionTok{sum}\NormalTok{( (x}\SpecialCharTok{{-}}\FunctionTok{mean}\NormalTok{(x))}\SpecialCharTok{\^{}}\DecValTok{2}\NormalTok{ )}
\NormalTok{b0 }\OtherTok{\textless{}{-}} \FunctionTok{mean}\NormalTok{(y) }\SpecialCharTok{{-}}\NormalTok{ b1}\SpecialCharTok{*}\FunctionTok{mean}\NormalTok{(x)}
\FunctionTok{c}\NormalTok{(}\AttributeTok{b0=}\NormalTok{b0, }\AttributeTok{b1=}\NormalTok{b1)}
\end{Highlighting}
\end{Shaded}

\begin{verbatim}
       b0        b1 
37.285126 -5.344472 
\end{verbatim}

\begin{center}\rule{0.5\linewidth}{0.5pt}\end{center}

\subsection{\texorpdfstring{Interpretação de
\(\hat\beta_1\)}{Interpretação de \textbackslash hat\textbackslash beta\_1}}\label{interpretauxe7uxe3o-de-hatbeta_1}

\begin{itemize}
\tightlist
\item
  \textbf{Sinal}: direção (positiva/negativa).
\item
  \textbf{Magnitude}: variação média de \(Y\) por 1 unidade de \(X\).
\end{itemize}

\begin{Shaded}
\begin{Highlighting}[]
\FunctionTok{summary}\NormalTok{(m\_slr)}\SpecialCharTok{$}\NormalTok{coefficients}
\end{Highlighting}
\end{Shaded}

\begin{verbatim}
             Estimate Std. Error   t value     Pr(>|t|)
(Intercept) 37.285126   1.877627 19.857575 8.241799e-19
wt          -5.344472   0.559101 -9.559044 1.293959e-10
\end{verbatim}

\begin{center}\rule{0.5\linewidth}{0.5pt}\end{center}

\subsection{Propriedades
(Gauss--Markov)}\label{propriedades-gaussmarkov}

\begin{itemize}
\tightlist
\item
  Sob \(\mathbb{E}[\varepsilon|X]=0\) e
  \(\mathrm{Var}(\varepsilon|X)=\sigma^2 I\).
\end{itemize}

\begin{Shaded}
\begin{Highlighting}[]
\CommentTok{\# Rápida inspeção dos resíduos}
\FunctionTok{par}\NormalTok{(}\AttributeTok{mfrow=}\FunctionTok{c}\NormalTok{(}\DecValTok{1}\NormalTok{,}\DecValTok{2}\NormalTok{))}
\FunctionTok{plot}\NormalTok{(m\_slr, }\AttributeTok{which=}\DecValTok{1}\NormalTok{)  }\CommentTok{\# resíduos vs ajustados}
\FunctionTok{plot}\NormalTok{(m\_slr, }\AttributeTok{which=}\DecValTok{2}\NormalTok{)  }\CommentTok{\# QQ{-}plot}
\end{Highlighting}
\end{Shaded}

\pandocbounded{\includegraphics[keepaspectratio]{Aula4_files/figure-pdf/unnamed-chunk-7-1.pdf}}

\begin{Shaded}
\begin{Highlighting}[]
\FunctionTok{par}\NormalTok{(}\AttributeTok{mfrow=}\FunctionTok{c}\NormalTok{(}\DecValTok{1}\NormalTok{,}\DecValTok{1}\NormalTok{))}
\end{Highlighting}
\end{Shaded}

\begin{center}\rule{0.5\linewidth}{0.5pt}\end{center}

\subsection{\texorpdfstring{Estimador de \(\sigma^2\) e
RSE}{Estimador de \textbackslash sigma\^{}2 e RSE}}\label{estimador-de-sigma2-e-rse}

\[
\hat\sigma^2=\frac{1}{n-2}\sum e_i^2,\quad \mathrm{RSE}=\sqrt{\hat\sigma^2}
\]

\begin{Shaded}
\begin{Highlighting}[]
\NormalTok{sigma2\_hat }\OtherTok{\textless{}{-}} \FunctionTok{sum}\NormalTok{(}\FunctionTok{residuals}\NormalTok{(m\_slr)}\SpecialCharTok{\^{}}\DecValTok{2}\NormalTok{)}\SpecialCharTok{/}\NormalTok{(n}\DecValTok{{-}2}\NormalTok{)}
\NormalTok{RSE }\OtherTok{\textless{}{-}} \FunctionTok{sqrt}\NormalTok{(sigma2\_hat)}
\FunctionTok{c}\NormalTok{(}\AttributeTok{sigma2\_hat=}\NormalTok{sigma2\_hat, }\AttributeTok{RSE=}\NormalTok{RSE)}
\end{Highlighting}
\end{Shaded}

\begin{verbatim}
sigma2_hat        RSE 
  9.277398   3.045882 
\end{verbatim}

\begin{center}\rule{0.5\linewidth}{0.5pt}\end{center}

\subsection{Erros-padrão dos
coeficientes}\label{erros-padruxe3o-dos-coeficientes}

\[
\mathrm{SE}(\hat\beta_1)=\frac{\hat\sigma}{\sqrt{\sum (X_i-\bar X)^2}},\qquad
\mathrm{SE}(\hat\beta_0)=\hat\sigma\sqrt{\frac{1}{n}+\frac{\bar X^2}{\sum (X_i-\bar X)^2}}
\]

\begin{Shaded}
\begin{Highlighting}[]
\CommentTok{\# Conferir com o summary()}
\FunctionTok{summary}\NormalTok{(m\_slr)}\SpecialCharTok{$}\NormalTok{coefficients[,}\DecValTok{2}\NormalTok{]}
\end{Highlighting}
\end{Shaded}

\begin{verbatim}
(Intercept)          wt 
   1.877627    0.559101 
\end{verbatim}

\begin{center}\rule{0.5\linewidth}{0.5pt}\end{center}

\subsection{\texorpdfstring{Testes \(t\) para
coeficientes}{Testes t para coeficientes}}\label{testes-t-para-coeficientes}

\begin{itemize}
\tightlist
\item
  \(H_0:\beta_1=0\) vs \(H_1:\beta_1\neq 0\)
\item
  Estatística: \(t=\hat\beta_1/\mathrm{SE}(\hat\beta_1)\) (gl \(n-2\))
\end{itemize}

\begin{Shaded}
\begin{Highlighting}[]
\CommentTok{\# Estatística t e p{-}valor para beta1}
\NormalTok{t\_val }\OtherTok{\textless{}{-}} \FunctionTok{summary}\NormalTok{(m\_slr)}\SpecialCharTok{$}\NormalTok{coefficients[}\StringTok{"wt"}\NormalTok{,}\StringTok{"t value"}\NormalTok{]}
\NormalTok{p\_val }\OtherTok{\textless{}{-}} \FunctionTok{summary}\NormalTok{(m\_slr)}\SpecialCharTok{$}\NormalTok{coefficients[}\StringTok{"wt"}\NormalTok{,}\StringTok{"Pr(\textgreater{}|t|)"}\NormalTok{]}
\FunctionTok{c}\NormalTok{(}\AttributeTok{t=}\NormalTok{t\_val, }\AttributeTok{p=}\NormalTok{p\_val)}
\end{Highlighting}
\end{Shaded}

\begin{verbatim}
            t             p 
-9.559044e+00  1.293959e-10 
\end{verbatim}

\begin{center}\rule{0.5\linewidth}{0.5pt}\end{center}

\subsection{ICs para coeficientes}\label{ics-para-coeficientes}

\begin{itemize}
\tightlist
\item
  IC \((1-\alpha)\):
  \(\hat\beta_j \pm t_{1-\alpha/2},\mathrm{SE}(\hat\beta_j)\).
\end{itemize}

\begin{Shaded}
\begin{Highlighting}[]
\FunctionTok{confint}\NormalTok{(m\_slr)}
\end{Highlighting}
\end{Shaded}

\begin{verbatim}
                2.5 %    97.5 %
(Intercept) 33.450500 41.119753
wt          -6.486308 -4.202635
\end{verbatim}

\begin{center}\rule{0.5\linewidth}{0.5pt}\end{center}

\subsection{\texorpdfstring{Medidas de ajuste:
\(R^2\)}{Medidas de ajuste: R\^{}2}}\label{medidas-de-ajuste-r2}

\[
R^2 = 1-\frac{\sum e_i^2}{\sum (Y_i-\bar Y)^2}
\]

\begin{Shaded}
\begin{Highlighting}[]
\FunctionTok{summary}\NormalTok{(m\_slr)}\SpecialCharTok{$}\NormalTok{r.squared}
\end{Highlighting}
\end{Shaded}

\begin{verbatim}
[1] 0.7528328
\end{verbatim}

\begin{center}\rule{0.5\linewidth}{0.5pt}\end{center}

\subsection{Previsão: ponto \textbar{} IC da média \textbar{} IP
individual}\label{previsuxe3o-ponto-ic-da-muxe9dia-ip-individual}

\begin{Shaded}
\begin{Highlighting}[]
\CommentTok{\# Ponto, IC da média e IP para wt=3}
\FunctionTok{predict}\NormalTok{(m\_slr, }\AttributeTok{newdata =} \FunctionTok{data.frame}\NormalTok{(}\AttributeTok{wt=}\DecValTok{3}\NormalTok{), }\AttributeTok{interval=}\StringTok{"confidence"}\NormalTok{)}
\end{Highlighting}
\end{Shaded}

\begin{verbatim}
       fit      lwr      upr
1 21.25171 20.12444 22.37899
\end{verbatim}

\begin{Shaded}
\begin{Highlighting}[]
\FunctionTok{predict}\NormalTok{(m\_slr, }\AttributeTok{newdata =} \FunctionTok{data.frame}\NormalTok{(}\AttributeTok{wt=}\DecValTok{3}\NormalTok{), }\AttributeTok{interval=}\StringTok{"prediction"}\NormalTok{)}
\end{Highlighting}
\end{Shaded}

\begin{verbatim}
       fit      lwr      upr
1 21.25171 14.92987 27.57355
\end{verbatim}

\begin{center}\rule{0.5\linewidth}{0.5pt}\end{center}

\subsection{Pressupostos: visão
prática}\label{pressupostos-visuxe3o-pruxe1tica}

\begin{itemize}
\tightlist
\item
  \textbf{Linearidade}, \textbf{homoscedasticidade},
  \textbf{independência}, \textbf{normalidade} (para inferência).
\item
  Verificar com gráficos de resíduos e QQ-plots.
\end{itemize}

\begin{center}\rule{0.5\linewidth}{0.5pt}\end{center}

\subsection{Diagnóstico inicial
(gráficos)}\label{diagnuxf3stico-inicial-gruxe1ficos}

\begin{Shaded}
\begin{Highlighting}[]
\FunctionTok{par}\NormalTok{(}\AttributeTok{mfrow=}\FunctionTok{c}\NormalTok{(}\DecValTok{2}\NormalTok{,}\DecValTok{2}\NormalTok{))}
\FunctionTok{plot}\NormalTok{(m\_slr)          }\CommentTok{\# 4 gráficos padrão de diagnóstico}
\end{Highlighting}
\end{Shaded}

\pandocbounded{\includegraphics[keepaspectratio]{Aula4_files/figure-pdf/unnamed-chunk-14-1.pdf}}

\begin{Shaded}
\begin{Highlighting}[]
\FunctionTok{par}\NormalTok{(}\AttributeTok{mfrow=}\FunctionTok{c}\NormalTok{(}\DecValTok{1}\NormalTok{,}\DecValTok{1}\NormalTok{))}
\end{Highlighting}
\end{Shaded}

\begin{center}\rule{0.5\linewidth}{0.5pt}\end{center}

\subsection{Quando SLR falha
(indícios)}\label{quando-slr-falha-induxedcios}

\begin{itemize}
\tightlist
\item
  Relação curvilínea, heteroscedasticidade, outliers/influência, omissão
  de preditores.
\end{itemize}

\begin{center}\rule{0.5\linewidth}{0.5pt}\end{center}

\subsection{Exercícios rápidos --- Aula
1}\label{exercuxedcios-ruxe1pidos-aula-1}

\begin{enumerate}
\def\labelenumi{\arabic{enumi}.}
\tightlist
\item
  \textbf{Ajusta e interpreta} \texttt{mpg\ \textasciitilde{}\ wt}.
\item
  \textbf{Testa} \(H_0!:\beta_1=0\) a 5\%.
\item
  \textbf{IC/IP} em \texttt{wt=2.2} e \texttt{wt=3.0}.
\item
  \textbf{Diagnóstico}: comenta resíduos vs ajustados e QQ-plot.
\end{enumerate}

\begin{center}\rule{0.5\linewidth}{0.5pt}\end{center}

\subsubsection{\texorpdfstring{Exercício final --- SLR com
\texttt{women} (altura →
peso)}{Exercício final --- SLR com women (altura → peso)}}\label{exercuxedcio-final-slr-com-women-altura-peso}

\textbf{Objetivo:} ajustar e interpretar um modelo de regressão linear
simples do \textbf{peso} (\texttt{weight}) em função da \textbf{altura}
(\texttt{height}). \textbf{Dados:} \texttt{women} (R base; 15
observações de mulheres adultas nos EUA).

\begin{center}\rule{0.5\linewidth}{0.5pt}\end{center}

\paragraph{Tarefas}\label{tarefas}

\begin{enumerate}
\def\labelenumi{\arabic{enumi}.}
\item
  \textbf{Inspeção inicial}

  1.1. Carrega \texttt{women}; mostra primeiras linhas e estatísticas
  resumo.

  1.2. Faz um gráfico de dispersão de \texttt{weight} vs \texttt{height}
  (peso no eixo Y, altura no X).

  1.3. Comenta visualmente se a relação parece aproximadamente linear.
\end{enumerate}

\begin{center}\rule{0.5\linewidth}{0.5pt}\end{center}

\begin{enumerate}
\def\labelenumi{\arabic{enumi}.}
\setcounter{enumi}{1}
\item
  \textbf{Ajuste do modelo SLR}

  2.1. Ajusta o modelo: \texttt{weight\ \textasciitilde{}\ height}.

  2.2. Escreve a reta estimada:
  \(\widehat{\text{weight}} = \hat\beta_0 + \hat\beta_1 \cdot \text{height}\).

  2.3. \textbf{Interpreta} \(\hat\beta_1\) (unidades e significado) e
  \(\hat\beta_0\) (faz sentido físico? comenta).
\end{enumerate}

\begin{center}\rule{0.5\linewidth}{0.5pt}\end{center}

\begin{enumerate}
\def\labelenumi{\arabic{enumi}.}
\setcounter{enumi}{2}
\item
  \textbf{Inferência}

  3.1. Testa \(H_0:\beta_1=0\) a 5\% (apresenta \(t\), p-valor e
  conclusão).

  3.2. Apresenta o \textbf{IC95\%} para \(\beta_1\).

  3.3. Reporta \textbf{RSE} e \textbf{(R\^{}2)}; comenta a qualidade de
  ajuste.
\end{enumerate}

\begin{center}\rule{0.5\linewidth}{0.5pt}\end{center}

\begin{enumerate}
\def\labelenumi{\arabic{enumi}.}
\setcounter{enumi}{3}
\item
  \textbf{Previsão}

  4.1. Obtém \textbf{previsão pontual}, \textbf{IC da média} e
  \textbf{IP individual} para \texttt{height\ =\ 66} e
  \texttt{height\ =\ 70}.

  4.2. Compara as larguras de IC vs IP e entre as duas alturas (mais
  longe da média → maior incerteza?).
\end{enumerate}

\begin{center}\rule{0.5\linewidth}{0.5pt}\end{center}

\begin{enumerate}
\def\labelenumi{\arabic{enumi}.}
\setcounter{enumi}{4}
\item
  \textbf{Diagnóstico}

  5.1. Mostra os 4 gráficos padrão (\texttt{plot(lm\_obj)}) e comenta:
  linearidade (resíduos vs ajustados) e normalidade (QQ-plot).

  5.2. Há alguma observação com \textbf{alta influência} (Cook)?
  Identifica-a e comenta o impacto potencial.
\end{enumerate}

\begin{center}\rule{0.5\linewidth}{0.5pt}\end{center}

\subsection{Referências}\label{referuxeancias}

\begin{itemize}
\tightlist
\item
  James, Witten, Hastie, Tibshirani (2023). \emph{An Introduction to
  Statistical Learning}, 2.ª ed., Cap. 3.
\item
  R base: \texttt{?lm}, \texttt{plot.lm}, \texttt{predict.lm},
  \texttt{anova.lm}.
\item
  Pacotes úteis: \texttt{car} (VIF, avPlots), \texttt{lmtest} (BP, DW),
  \texttt{sandwich} (SE robustos).
\end{itemize}




\end{document}
