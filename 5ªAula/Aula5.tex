% Options for packages loaded elsewhere
% Options for packages loaded elsewhere
\PassOptionsToPackage{unicode}{hyperref}
\PassOptionsToPackage{hyphens}{url}
\PassOptionsToPackage{dvipsnames,svgnames,x11names}{xcolor}
%
\documentclass[
  portuguese,
  letterpaper,
  DIV=11,
  numbers=noendperiod]{scrartcl}
\usepackage{xcolor}
\usepackage{amsmath,amssymb}
\setcounter{secnumdepth}{-\maxdimen} % remove section numbering
\usepackage{iftex}
\ifPDFTeX
  \usepackage[T1]{fontenc}
  \usepackage[utf8]{inputenc}
  \usepackage{textcomp} % provide euro and other symbols
\else % if luatex or xetex
  \usepackage{unicode-math} % this also loads fontspec
  \defaultfontfeatures{Scale=MatchLowercase}
  \defaultfontfeatures[\rmfamily]{Ligatures=TeX,Scale=1}
\fi
\usepackage{lmodern}
\ifPDFTeX\else
  % xetex/luatex font selection
\fi
% Use upquote if available, for straight quotes in verbatim environments
\IfFileExists{upquote.sty}{\usepackage{upquote}}{}
\IfFileExists{microtype.sty}{% use microtype if available
  \usepackage[]{microtype}
  \UseMicrotypeSet[protrusion]{basicmath} % disable protrusion for tt fonts
}{}
\makeatletter
\@ifundefined{KOMAClassName}{% if non-KOMA class
  \IfFileExists{parskip.sty}{%
    \usepackage{parskip}
  }{% else
    \setlength{\parindent}{0pt}
    \setlength{\parskip}{6pt plus 2pt minus 1pt}}
}{% if KOMA class
  \KOMAoptions{parskip=half}}
\makeatother
% Make \paragraph and \subparagraph free-standing
\makeatletter
\ifx\paragraph\undefined\else
  \let\oldparagraph\paragraph
  \renewcommand{\paragraph}{
    \@ifstar
      \xxxParagraphStar
      \xxxParagraphNoStar
  }
  \newcommand{\xxxParagraphStar}[1]{\oldparagraph*{#1}\mbox{}}
  \newcommand{\xxxParagraphNoStar}[1]{\oldparagraph{#1}\mbox{}}
\fi
\ifx\subparagraph\undefined\else
  \let\oldsubparagraph\subparagraph
  \renewcommand{\subparagraph}{
    \@ifstar
      \xxxSubParagraphStar
      \xxxSubParagraphNoStar
  }
  \newcommand{\xxxSubParagraphStar}[1]{\oldsubparagraph*{#1}\mbox{}}
  \newcommand{\xxxSubParagraphNoStar}[1]{\oldsubparagraph{#1}\mbox{}}
\fi
\makeatother

\usepackage{color}
\usepackage{fancyvrb}
\newcommand{\VerbBar}{|}
\newcommand{\VERB}{\Verb[commandchars=\\\{\}]}
\DefineVerbatimEnvironment{Highlighting}{Verbatim}{commandchars=\\\{\}}
% Add ',fontsize=\small' for more characters per line
\usepackage{framed}
\definecolor{shadecolor}{RGB}{241,243,245}
\newenvironment{Shaded}{\begin{snugshade}}{\end{snugshade}}
\newcommand{\AlertTok}[1]{\textcolor[rgb]{0.68,0.00,0.00}{#1}}
\newcommand{\AnnotationTok}[1]{\textcolor[rgb]{0.37,0.37,0.37}{#1}}
\newcommand{\AttributeTok}[1]{\textcolor[rgb]{0.40,0.45,0.13}{#1}}
\newcommand{\BaseNTok}[1]{\textcolor[rgb]{0.68,0.00,0.00}{#1}}
\newcommand{\BuiltInTok}[1]{\textcolor[rgb]{0.00,0.23,0.31}{#1}}
\newcommand{\CharTok}[1]{\textcolor[rgb]{0.13,0.47,0.30}{#1}}
\newcommand{\CommentTok}[1]{\textcolor[rgb]{0.37,0.37,0.37}{#1}}
\newcommand{\CommentVarTok}[1]{\textcolor[rgb]{0.37,0.37,0.37}{\textit{#1}}}
\newcommand{\ConstantTok}[1]{\textcolor[rgb]{0.56,0.35,0.01}{#1}}
\newcommand{\ControlFlowTok}[1]{\textcolor[rgb]{0.00,0.23,0.31}{\textbf{#1}}}
\newcommand{\DataTypeTok}[1]{\textcolor[rgb]{0.68,0.00,0.00}{#1}}
\newcommand{\DecValTok}[1]{\textcolor[rgb]{0.68,0.00,0.00}{#1}}
\newcommand{\DocumentationTok}[1]{\textcolor[rgb]{0.37,0.37,0.37}{\textit{#1}}}
\newcommand{\ErrorTok}[1]{\textcolor[rgb]{0.68,0.00,0.00}{#1}}
\newcommand{\ExtensionTok}[1]{\textcolor[rgb]{0.00,0.23,0.31}{#1}}
\newcommand{\FloatTok}[1]{\textcolor[rgb]{0.68,0.00,0.00}{#1}}
\newcommand{\FunctionTok}[1]{\textcolor[rgb]{0.28,0.35,0.67}{#1}}
\newcommand{\ImportTok}[1]{\textcolor[rgb]{0.00,0.46,0.62}{#1}}
\newcommand{\InformationTok}[1]{\textcolor[rgb]{0.37,0.37,0.37}{#1}}
\newcommand{\KeywordTok}[1]{\textcolor[rgb]{0.00,0.23,0.31}{\textbf{#1}}}
\newcommand{\NormalTok}[1]{\textcolor[rgb]{0.00,0.23,0.31}{#1}}
\newcommand{\OperatorTok}[1]{\textcolor[rgb]{0.37,0.37,0.37}{#1}}
\newcommand{\OtherTok}[1]{\textcolor[rgb]{0.00,0.23,0.31}{#1}}
\newcommand{\PreprocessorTok}[1]{\textcolor[rgb]{0.68,0.00,0.00}{#1}}
\newcommand{\RegionMarkerTok}[1]{\textcolor[rgb]{0.00,0.23,0.31}{#1}}
\newcommand{\SpecialCharTok}[1]{\textcolor[rgb]{0.37,0.37,0.37}{#1}}
\newcommand{\SpecialStringTok}[1]{\textcolor[rgb]{0.13,0.47,0.30}{#1}}
\newcommand{\StringTok}[1]{\textcolor[rgb]{0.13,0.47,0.30}{#1}}
\newcommand{\VariableTok}[1]{\textcolor[rgb]{0.07,0.07,0.07}{#1}}
\newcommand{\VerbatimStringTok}[1]{\textcolor[rgb]{0.13,0.47,0.30}{#1}}
\newcommand{\WarningTok}[1]{\textcolor[rgb]{0.37,0.37,0.37}{\textit{#1}}}

\usepackage{longtable,booktabs,array}
\usepackage{calc} % for calculating minipage widths
% Correct order of tables after \paragraph or \subparagraph
\usepackage{etoolbox}
\makeatletter
\patchcmd\longtable{\par}{\if@noskipsec\mbox{}\fi\par}{}{}
\makeatother
% Allow footnotes in longtable head/foot
\IfFileExists{footnotehyper.sty}{\usepackage{footnotehyper}}{\usepackage{footnote}}
\makesavenoteenv{longtable}
\usepackage{graphicx}
\makeatletter
\newsavebox\pandoc@box
\newcommand*\pandocbounded[1]{% scales image to fit in text height/width
  \sbox\pandoc@box{#1}%
  \Gscale@div\@tempa{\textheight}{\dimexpr\ht\pandoc@box+\dp\pandoc@box\relax}%
  \Gscale@div\@tempb{\linewidth}{\wd\pandoc@box}%
  \ifdim\@tempb\p@<\@tempa\p@\let\@tempa\@tempb\fi% select the smaller of both
  \ifdim\@tempa\p@<\p@\scalebox{\@tempa}{\usebox\pandoc@box}%
  \else\usebox{\pandoc@box}%
  \fi%
}
% Set default figure placement to htbp
\def\fps@figure{htbp}
\makeatother



\ifLuaTeX
\usepackage[bidi=basic]{babel}
\else
\usepackage[bidi=default]{babel}
\fi
% get rid of language-specific shorthands (see #6817):
\let\LanguageShortHands\languageshorthands
\def\languageshorthands#1{}


\setlength{\emergencystretch}{3em} % prevent overfull lines

\providecommand{\tightlist}{%
  \setlength{\itemsep}{0pt}\setlength{\parskip}{0pt}}



 


\KOMAoption{captions}{tableheading}
\makeatletter
\@ifpackageloaded{caption}{}{\usepackage{caption}}
\AtBeginDocument{%
\ifdefined\contentsname
  \renewcommand*\contentsname{Índice}
\else
  \newcommand\contentsname{Índice}
\fi
\ifdefined\listfigurename
  \renewcommand*\listfigurename{Lista de Figuras}
\else
  \newcommand\listfigurename{Lista de Figuras}
\fi
\ifdefined\listtablename
  \renewcommand*\listtablename{Lista de Tabelas}
\else
  \newcommand\listtablename{Lista de Tabelas}
\fi
\ifdefined\figurename
  \renewcommand*\figurename{Figura}
\else
  \newcommand\figurename{Figura}
\fi
\ifdefined\tablename
  \renewcommand*\tablename{Tabela}
\else
  \newcommand\tablename{Tabela}
\fi
}
\@ifpackageloaded{float}{}{\usepackage{float}}
\floatstyle{ruled}
\@ifundefined{c@chapter}{\newfloat{codelisting}{h}{lop}}{\newfloat{codelisting}{h}{lop}[chapter]}
\floatname{codelisting}{Listagem}
\newcommand*\listoflistings{\listof{codelisting}{Lista de Listagens}}
\makeatother
\makeatletter
\makeatother
\makeatletter
\@ifpackageloaded{caption}{}{\usepackage{caption}}
\@ifpackageloaded{subcaption}{}{\usepackage{subcaption}}
\makeatother
\usepackage{bookmark}
\IfFileExists{xurl.sty}{\usepackage{xurl}}{} % add URL line breaks if available
\urlstyle{same}
\hypersetup{
  pdftitle={Regressão Linear Múltipla},
  pdflang={pt},
  colorlinks=true,
  linkcolor={blue},
  filecolor={Maroon},
  citecolor={Blue},
  urlcolor={Blue},
  pdfcreator={LaTeX via pandoc}}


\title{Regressão Linear Múltipla}
\author{}
\date{2026-04-10}
\begin{document}
\maketitle


\begin{center}\rule{0.5\linewidth}{0.5pt}\end{center}

\subsection{\texorpdfstring{Porquê regressão linear
\textbf{múltipla}?}{Porquê regressão linear múltipla?}}\label{porquuxea-regressuxe3o-linear-muxfaltipla}

\begin{itemize}
\tightlist
\item
  \textbf{Fenómenos multicausais}: fenómenos reais raramente dependem de
  uma única variável.
\item
  \textbf{Melhor predição}: ao considerar variáveis relevantes reduzimos
  erro de predição.
\item
  \textbf{Separar efeitos}: pretendemos isolar o impacto de cada
  variável \textbf{mantendo as outras constantes}.
\item
  \textbf{Interações}: o efeito de uma variável pode \textbf{depender}
  do nível de outra (ex.: efeito do treino depende da dieta).
\end{itemize}

\begin{center}\rule{0.5\linewidth}{0.5pt}\end{center}

\subsection{Modelo: forma escalar e
matricial}\label{modelo-forma-escalar-e-matricial}

Dados \((y_i, x_{i1},\dots,x_{ip})\), \(i=1,\dots,n\): \[
y_i \;=\; \beta_0 + \beta_1 x_{i1} + \cdots + \beta_p x_{ip} + \varepsilon_i,
\]

\(\mathbb E[\varepsilon_i]=0,\;
\mathrm{Var}(\varepsilon_i)=\sigma^2,\;
\mathrm{Cov}(\varepsilon)=0.\)

\hfill\break

Notação matricial: \(\mathbf y = X\beta + \varepsilon\),
\(\varepsilon \sim (0,\sigma^2 I)\).

\begin{center}\rule{0.5\linewidth}{0.5pt}\end{center}

\textbf{MMQ (OLS)}: \(\hat\beta = (X^\top X)^{-1}X^\top \mathbf y\).\\
Interpretação de \(\beta_j\): variação \textbf{média} em \(Y\) por
\textbf{unidade} de \(X_j\), \textbf{mantendo} as restantes fixas.

O estimador MMQ \[
  \hat\beta=(X^\top X)^{-1}X^\top y
  \] só existe (e é único) se \(X^\top X\) for invertível ⇒ colunas de
\(X\) \textbf{linearmente independentes}.

\begin{center}\rule{0.5\linewidth}{0.5pt}\end{center}

\subsection{\texorpdfstring{Pressupostos sobre os
\textbf{erros}}{Pressupostos sobre os erros}}\label{pressupostos-sobre-os-erros}

\begin{enumerate}
\def\labelenumi{\arabic{enumi})}
\item
  \textbf{Média zero}\\
  \(E[\varepsilon\mid X]=0\)
\item
  \textbf{Homocedasticidade (variância constante)}\\
  \(\operatorname{Var}(\varepsilon\mid X)=\sigma^2\)
\item
  \textbf{Independência / ausência de autocorrelação}\\
  \(\operatorname{Cov}(\varepsilon_i,\varepsilon_j\mid X)=0\ (i\neq j)\)
\end{enumerate}

\begin{center}\rule{0.5\linewidth}{0.5pt}\end{center}

\subsection{O que esperar dos resíduos se OLS é
adequado}\label{o-que-esperar-dos-resuxedduos-se-ols-uxe9-adequado}

\begin{itemize}
\tightlist
\item
  \(\sum_i \hat{\varepsilon}_i=0\) (modelo com intercept).\\
\item
  \(X^\top \hat{\varepsilon} = 0\) (resíduos ortogonais aos
  regressores).\\
\item
  Resíduos vs ajustados: \textbf{sem padrão} e variância aproximadamente
  constante.\\
\item
  QQ-plot: pontos próximos da reta (se normalidade).
\end{itemize}

\subsection{\texorpdfstring{Exemplo A : \texttt{mtcars} --- consumo vs
massa, potência e
caixa}{Exemplo A : mtcars --- consumo vs massa, potência e caixa}}\label{exemplo-a-mtcars-consumo-vs-massa-potuxeancia-e-caixa}

\begin{Shaded}
\begin{Highlighting}[]
\FunctionTok{data}\NormalTok{(mtcars)}
\FunctionTok{library}\NormalTok{(tidyverse)}
\FunctionTok{library}\NormalTok{(broom)}
\NormalTok{mt }\OtherTok{\textless{}{-}}\NormalTok{ mtcars }\SpecialCharTok{|\textgreater{}}
  \FunctionTok{mutate}\NormalTok{(}\AttributeTok{am =} \FunctionTok{factor}\NormalTok{(am, }\AttributeTok{labels =} \FunctionTok{c}\NormalTok{(}\StringTok{"auto"}\NormalTok{,}\StringTok{"manual"}\NormalTok{)),}
         \AttributeTok{wt\_kg =}\NormalTok{ wt }\SpecialCharTok{*} \FloatTok{453.592} \SpecialCharTok{/} \DecValTok{1000}\NormalTok{)  }\CommentTok{\# wt em milhares de libras; converter para kg}
\NormalTok{modA }\OtherTok{\textless{}{-}} \FunctionTok{lm}\NormalTok{(mpg }\SpecialCharTok{\textasciitilde{}}\NormalTok{ wt\_kg }\SpecialCharTok{+}\NormalTok{ hp }\SpecialCharTok{+}\NormalTok{ am, }\AttributeTok{data =}\NormalTok{ mt)}
\FunctionTok{summary}\NormalTok{(modA)}
\end{Highlighting}
\end{Shaded}

\begin{verbatim}

Call:
lm(formula = mpg ~ wt_kg + hp + am, data = mt)

Residuals:
    Min      1Q  Median      3Q     Max 
-3.4221 -1.7924 -0.3788  1.2249  5.5317 

Coefficients:
             Estimate Std. Error t value Pr(>|t|)    
(Intercept) 34.002875   2.642659  12.867 2.82e-13 ***
wt_kg       -6.346178   1.995120  -3.181 0.003574 ** 
hp          -0.037479   0.009605  -3.902 0.000546 ***
ammanual     2.083710   1.376420   1.514 0.141268    
---
Signif. codes:  0 '***' 0.001 '**' 0.01 '*' 0.05 '.' 0.1 ' ' 1

Residual standard error: 2.538 on 28 degrees of freedom
Multiple R-squared:  0.8399,    Adjusted R-squared:  0.8227 
F-statistic: 48.96 on 3 and 28 DF,  p-value: 2.908e-11
\end{verbatim}

\begin{Shaded}
\begin{Highlighting}[]
\FunctionTok{tidy}\NormalTok{(modA, }\AttributeTok{conf.int =} \ConstantTok{TRUE}\NormalTok{)}
\end{Highlighting}
\end{Shaded}

\begin{verbatim}
# A tibble: 4 x 7
  term        estimate std.error statistic  p.value conf.low conf.high
  <chr>          <dbl>     <dbl>     <dbl>    <dbl>    <dbl>     <dbl>
1 (Intercept)  34.0      2.64        12.9  2.82e-13  28.6      39.4   
2 wt_kg        -6.35     2.00        -3.18 3.57e- 3 -10.4      -2.26  
3 hp           -0.0375   0.00961     -3.90 5.46e- 4  -0.0572   -0.0178
4 ammanual      2.08     1.38         1.51 1.41e- 1  -0.736     4.90  
\end{verbatim}

\begin{Shaded}
\begin{Highlighting}[]
\FunctionTok{glance}\NormalTok{(modA)[, }\FunctionTok{c}\NormalTok{(}\StringTok{"r.squared"}\NormalTok{,}\StringTok{"adj.r.squared"}\NormalTok{,}\StringTok{"sigma"}\NormalTok{)]}
\end{Highlighting}
\end{Shaded}

\begin{verbatim}
# A tibble: 1 x 3
  r.squared adj.r.squared sigma
      <dbl>         <dbl> <dbl>
1     0.840         0.823  2.54
\end{verbatim}

\textbf{Interpretação.}\\
- \textbf{wt\_kg}: variação média de \textbf{mpg} por +1 kg,
\textbf{controlando} hp e am (espera-se coeficiente negativo).\\
- \textbf{hp}: efeito parcial de potência (mpg ↓ quando hp ↑, dado wt e
am fixos).\\
- \textbf{am}: diferença média entre \textbf{manual} e \textbf{auto},
para os mesmos wt\_kg e hp.\\
- \(R^2\) ajustado e \(\hat\sigma\) medem qualidade do ajuste e
dispersão residual.

\begin{center}\rule{0.5\linewidth}{0.5pt}\end{center}

\subsection{Efeitos marginais e previsões com
incerteza}\label{efeitos-marginais-e-previsuxf5es-com-incerteza}

\begin{Shaded}
\begin{Highlighting}[]
\CommentTok{\# Grella para explorar efeito de wt\_kg mantendo hp, am fixos}
\NormalTok{grid\_wt }\OtherTok{\textless{}{-}}\NormalTok{ mt }\SpecialCharTok{|\textgreater{}}
  \FunctionTok{summarize}\NormalTok{(}\AttributeTok{hp =} \FunctionTok{median}\NormalTok{(hp), }\AttributeTok{am =}\NormalTok{ am[}\DecValTok{1}\NormalTok{]) }\SpecialCharTok{|\textgreater{}}
  \FunctionTok{crossing}\NormalTok{(}\AttributeTok{wt\_kg =} \FunctionTok{seq}\NormalTok{(}\FunctionTok{min}\NormalTok{(mt}\SpecialCharTok{$}\NormalTok{wt\_kg), }\FunctionTok{max}\NormalTok{(mt}\SpecialCharTok{$}\NormalTok{wt\_kg), }\AttributeTok{length.out =} \DecValTok{50}\NormalTok{))}

\NormalTok{pred }\OtherTok{\textless{}{-}} \FunctionTok{predict}\NormalTok{(modA, }\AttributeTok{newdata =}\NormalTok{ grid\_wt, }\AttributeTok{interval =} \StringTok{"confidence"}\NormalTok{)}
\NormalTok{pred\_p }\OtherTok{\textless{}{-}} \FunctionTok{predict}\NormalTok{(modA, }\AttributeTok{newdata =}\NormalTok{ grid\_wt, }\AttributeTok{interval =} \StringTok{"prediction"}\NormalTok{)}

\FunctionTok{bind\_cols}\NormalTok{(grid\_wt, }\FunctionTok{as\_tibble}\NormalTok{(pred), }\FunctionTok{tibble}\NormalTok{(}\AttributeTok{lwr\_pred =}\NormalTok{ pred\_p[,}\DecValTok{3}\NormalTok{], }\AttributeTok{upr\_pred =}\NormalTok{ pred\_p[,}\DecValTok{3}\NormalTok{])) }\SpecialCharTok{|\textgreater{}} \FunctionTok{head}\NormalTok{()}
\end{Highlighting}
\end{Shaded}

\begin{verbatim}
# A tibble: 6 x 8
     hp am     wt_kg   fit   lwr   upr lwr_pred upr_pred
  <dbl> <fct>  <dbl> <dbl> <dbl> <dbl>    <dbl>    <dbl>
1   123 manual 0.686  27.1  25.0  29.3     32.8     32.8
2   123 manual 0.722  26.9  24.8  28.9     32.5     32.5
3   123 manual 0.759  26.7  24.7  28.6     32.2     32.2
4   123 manual 0.795  26.4  24.6  28.3     32.0     32.0
5   123 manual 0.831  26.2  24.4  28.0     31.7     31.7
6   123 manual 0.867  26.0  24.3  27.7     31.4     31.4
\end{verbatim}

\begin{center}\rule{0.5\linewidth}{0.5pt}\end{center}

\begin{Shaded}
\begin{Highlighting}[]
\NormalTok{pred\_df }\OtherTok{\textless{}{-}} \FunctionTok{bind\_cols}\NormalTok{(grid\_wt, }\FunctionTok{as\_tibble}\NormalTok{(pred)) }\SpecialCharTok{|\textgreater{}}
  \FunctionTok{rename}\NormalTok{(}\AttributeTok{fit=}\NormalTok{fit, }\AttributeTok{lwr=}\NormalTok{lwr, }\AttributeTok{upr=}\NormalTok{upr)}

\FunctionTok{ggplot}\NormalTok{(pred\_df, }\FunctionTok{aes}\NormalTok{(wt\_kg, fit)) }\SpecialCharTok{+}
  \FunctionTok{geom\_ribbon}\NormalTok{(}\FunctionTok{aes}\NormalTok{(}\AttributeTok{ymin=}\NormalTok{lwr, }\AttributeTok{ymax=}\NormalTok{upr), }\AttributeTok{alpha=}\NormalTok{.}\DecValTok{2}\NormalTok{) }\SpecialCharTok{+}
  \FunctionTok{geom\_line}\NormalTok{(}\AttributeTok{linewidth=}\DecValTok{1}\NormalTok{) }\SpecialCharTok{+}
  \FunctionTok{labs}\NormalTok{(}\AttributeTok{x=}\StringTok{"Massa (kg)"}\NormalTok{, }\AttributeTok{y=}\StringTok{"mpg (média prevista)"}\NormalTok{,}
       \AttributeTok{title=}\StringTok{"Efeito marginal de wt\_kg com IC da média (95\%) — hp e am fixos"}\NormalTok{)}
\end{Highlighting}
\end{Shaded}

\pandocbounded{\includegraphics[keepaspectratio]{Aula5_files/figure-pdf/unnamed-chunk-3-1.pdf}}

\textbf{Leitura.} Banda é \textbf{IC da média}; intervalo de
\textbf{predição} seria mais largo (variação individual).

\begin{center}\rule{0.5\linewidth}{0.5pt}\end{center}

\subsection{Diagnóstico e robustez: resíduos, QQ, distância de
Cook}\label{diagnuxf3stico-e-robustez-resuxedduos-qq-distuxe2ncia-de-cook}

\begin{Shaded}
\begin{Highlighting}[]
\FunctionTok{par}\NormalTok{(}\AttributeTok{mfrow=}\FunctionTok{c}\NormalTok{(}\DecValTok{1}\NormalTok{,}\DecValTok{3}\NormalTok{))}
\FunctionTok{plot}\NormalTok{(modA, }\AttributeTok{which =} \DecValTok{1}\NormalTok{)}
\FunctionTok{plot}\NormalTok{(modA, }\AttributeTok{which =} \DecValTok{2}\NormalTok{)}
\FunctionTok{plot}\NormalTok{(modA, }\AttributeTok{which =} \DecValTok{5}\NormalTok{)}
\end{Highlighting}
\end{Shaded}

\pandocbounded{\includegraphics[keepaspectratio]{Aula5_files/figure-pdf/unnamed-chunk-4-1.pdf}}

\begin{Shaded}
\begin{Highlighting}[]
\FunctionTok{par}\NormalTok{(}\AttributeTok{mfrow=}\FunctionTok{c}\NormalTok{(}\DecValTok{1}\NormalTok{,}\DecValTok{1}\NormalTok{))}
\end{Highlighting}
\end{Shaded}

\begin{center}\rule{0.5\linewidth}{0.5pt}\end{center}

\subsection{Multicolinearidade: VIF e
interpretação}\label{multicolinearidade-vif-e-interpretauxe7uxe3o}

\begin{itemize}
\item
  VIF (Variance Inflation Factor) mede quanto a \textbf{variância} de
  \(\hat\beta_j\) é \textbf{inflacionada} pela colinearidade de \(X_j\)
  com os restantes.
\item
  Definição via regressão auxiliar de \(X_j\) nas outras preditoras
  \(X_{-j}\): \[
  \text{VIF}_j \;=\; \frac{1}{1-R_j^2},
  \qquad
  R_j^2 \;=\; R^2\Big(\,X_j \sim X_{-j}\,\Big).
   \] ---
\item
  Relação com a variância de \(\hat\beta_j\): \[
  \operatorname{Var}(\hat\beta_j)
  \;=\;
  \sigma^2\,\frac{\text{VIF}_j}{\sum_i (x_{ij}-\bar x_j)^2}.
   \]
\end{itemize}

\begin{Shaded}
\begin{Highlighting}[]
\FunctionTok{library}\NormalTok{(car)}

\NormalTok{mod }\OtherTok{\textless{}{-}} \FunctionTok{lm}\NormalTok{(mpg }\SpecialCharTok{\textasciitilde{}}\NormalTok{ wt }\SpecialCharTok{+}\NormalTok{ hp, }\AttributeTok{data =}\NormalTok{ mtcars)}
\NormalTok{car}\SpecialCharTok{::}\FunctionTok{vif}\NormalTok{(mod)}
\end{Highlighting}
\end{Shaded}

\begin{verbatim}
      wt       hp 
1.766625 1.766625 
\end{verbatim}

\begin{center}\rule{0.5\linewidth}{0.5pt}\end{center}

\begin{itemize}
\tightlist
\item
  \textbf{Regra prática:}

  \begin{itemize}
  \tightlist
  \item
    VIF ≈ 1: sem colinearidade relevante\\
  \item
    VIF ≳ 5 (às vezes 10): \textbf{preocupante} →
    investigar/reparametrizar
  \end{itemize}
\item
  \textbf{Efeito:} VIF alto ⇒ \textbf{erros-padrão grandes}, ICs
  \textbf{largos}, sinais/valores de \(\hat\beta\) podem oscilar com
  pequenas mudanças nos dados.
\end{itemize}

\begin{center}\rule{0.5\linewidth}{0.5pt}\end{center}

\subsection{Fatores}\label{fatores}

\begin{itemize}
\item
  Em \texttt{lm(mpg\ \textasciitilde{}\ wt\_kg\ +\ hp\ +\ am)}, onde
  \texttt{am} é \textbf{fator} com níveis \texttt{"auto"} e
  \texttt{"manual"}, o R cria \textbf{variáveis indicadoras} (0/1) para
  representar os níveis.

  → Um nível fica como \textbf{base} (escolhido automaticamente,
  tipicamente o primeiro); os restantes têm um \textbf{coeficiente} que
  mede o desvio face a essa base.
\item
  \textbf{Como ler os coeficientes} (sem interações):

  \begin{itemize}
  \tightlist
  \item
    \texttt{ammanual}: \textbf{diferença média} (\emph{manual − base})
    em \texttt{mpg}, \textbf{mantendo} \texttt{wt\_kg} e \texttt{hp}
    fixos.\\
    Ex.: \texttt{ammanual\ =\ 2.5} ⇒ carros \emph{manuais} têm
    \textbf{+2.5 mpg} face à base para o mesmo peso e potência.
  \end{itemize}
\end{itemize}

\begin{Shaded}
\begin{Highlighting}[]
\FunctionTok{coef}\NormalTok{(}\FunctionTok{lm}\NormalTok{(mpg }\SpecialCharTok{\textasciitilde{}}\NormalTok{ wt\_kg }\SpecialCharTok{+}\NormalTok{ hp }\SpecialCharTok{+}\NormalTok{ am, }\AttributeTok{data =}\NormalTok{ mt))}
\end{Highlighting}
\end{Shaded}

\begin{verbatim}
(Intercept)       wt_kg          hp    ammanual 
34.00287512 -6.34617765 -0.03747873  2.08371013 
\end{verbatim}

\begin{center}\rule{0.5\linewidth}{0.5pt}\end{center}

\subsection{Interações: quando o efeito depende de outra
variável}\label{interauxe7uxf5es-quando-o-efeito-depende-de-outra-variuxe1vel}

\begin{Shaded}
\begin{Highlighting}[]
\NormalTok{modA\_int }\OtherTok{\textless{}{-}} \FunctionTok{lm}\NormalTok{(mpg }\SpecialCharTok{\textasciitilde{}}\NormalTok{ wt\_kg }\SpecialCharTok{*}\NormalTok{ am }\SpecialCharTok{+}\NormalTok{ hp, }\AttributeTok{data =}\NormalTok{ mt)}
\FunctionTok{tidy}\NormalTok{(modA\_int, }\AttributeTok{conf.int =} \ConstantTok{TRUE}\NormalTok{)}
\end{Highlighting}
\end{Shaded}

\begin{verbatim}
# A tibble: 5 x 7
  term           estimate std.error statistic  p.value conf.low conf.high
  <chr>             <dbl>     <dbl>     <dbl>    <dbl>    <dbl>     <dbl>
1 (Intercept)     30.9      2.72        11.4  8.55e-12  25.4     36.5    
2 wt_kg           -5.55     1.86        -2.98 6.05e- 3  -9.37    -1.73   
3 ammanual        11.6      4.02         2.87 7.84e- 3   3.30    19.8    
4 hp              -0.0269   0.00980     -2.75 1.05e- 2  -0.0470  -0.00685
5 wt_kg:ammanual  -7.89     3.18        -2.48 1.97e- 2 -14.4     -1.36   
\end{verbatim}

\begin{center}\rule{0.5\linewidth}{0.5pt}\end{center}

\begin{Shaded}
\begin{Highlighting}[]
\CommentTok{\# Visualizar duas retas (auto vs manual) em função de wt\_kg}
\NormalTok{new\_auto   }\OtherTok{\textless{}{-}} \FunctionTok{tibble}\NormalTok{(}\AttributeTok{wt\_kg =} \FunctionTok{seq}\NormalTok{(}\FunctionTok{min}\NormalTok{(mt}\SpecialCharTok{$}\NormalTok{wt\_kg), }\FunctionTok{max}\NormalTok{(mt}\SpecialCharTok{$}\NormalTok{wt\_kg), }\AttributeTok{length.out=}\DecValTok{50}\NormalTok{),}
                     \AttributeTok{hp =} \FunctionTok{median}\NormalTok{(mt}\SpecialCharTok{$}\NormalTok{hp), }\AttributeTok{am =} \FunctionTok{factor}\NormalTok{(}\StringTok{"auto"}\NormalTok{, }\AttributeTok{levels=}\FunctionTok{levels}\NormalTok{(mt}\SpecialCharTok{$}\NormalTok{am)))}
\NormalTok{new\_manual }\OtherTok{\textless{}{-}}\NormalTok{ new\_auto; new\_manual}\SpecialCharTok{$}\NormalTok{am }\OtherTok{\textless{}{-}} \FunctionTok{factor}\NormalTok{(}\StringTok{"manual"}\NormalTok{, }\AttributeTok{levels=}\FunctionTok{levels}\NormalTok{(mt}\SpecialCharTok{$}\NormalTok{am))}

\NormalTok{pred\_auto   }\OtherTok{\textless{}{-}} \FunctionTok{bind\_cols}\NormalTok{(new\_auto,   }\FunctionTok{as\_tibble}\NormalTok{(}\FunctionTok{predict}\NormalTok{(modA\_int, new\_auto, }\AttributeTok{interval=}\StringTok{"confidence"}\NormalTok{)))}
\NormalTok{pred\_manual }\OtherTok{\textless{}{-}} \FunctionTok{bind\_cols}\NormalTok{(new\_manual, }\FunctionTok{as\_tibble}\NormalTok{(}\FunctionTok{predict}\NormalTok{(modA\_int, new\_manual, }\AttributeTok{interval=}\StringTok{"confidence"}\NormalTok{)))}

\FunctionTok{ggplot}\NormalTok{() }\SpecialCharTok{+}
  \FunctionTok{geom\_ribbon}\NormalTok{(}\AttributeTok{data=}\NormalTok{pred\_auto, }\FunctionTok{aes}\NormalTok{(wt\_kg, }\AttributeTok{ymin=}\NormalTok{lwr, }\AttributeTok{ymax=}\NormalTok{upr), }\AttributeTok{alpha=}\NormalTok{.}\DecValTok{15}\NormalTok{) }\SpecialCharTok{+}
  \FunctionTok{geom\_line}\NormalTok{(}\AttributeTok{data=}\NormalTok{pred\_auto, }\FunctionTok{aes}\NormalTok{(wt\_kg, fit), }\AttributeTok{size=}\DecValTok{1}\NormalTok{) }\SpecialCharTok{+}
  \FunctionTok{geom\_ribbon}\NormalTok{(}\AttributeTok{data=}\NormalTok{pred\_manual, }\FunctionTok{aes}\NormalTok{(wt\_kg, }\AttributeTok{ymin=}\NormalTok{lwr, }\AttributeTok{ymax=}\NormalTok{upr), }\AttributeTok{alpha=}\NormalTok{.}\DecValTok{15}\NormalTok{) }\SpecialCharTok{+}
  \FunctionTok{geom\_line}\NormalTok{(}\AttributeTok{data=}\NormalTok{pred\_manual, }\FunctionTok{aes}\NormalTok{(wt\_kg, fit), }\AttributeTok{size=}\DecValTok{1}\NormalTok{) }\SpecialCharTok{+}
  \FunctionTok{labs}\NormalTok{(}\AttributeTok{x=}\StringTok{"Massa (kg)"}\NormalTok{, }\AttributeTok{y=}\StringTok{"mpg previsto"}\NormalTok{,}
       \AttributeTok{title=}\StringTok{"Interação wt\_kg × am: inclinações distintas por tipo de caixa"}\NormalTok{)}
\end{Highlighting}
\end{Shaded}

\pandocbounded{\includegraphics[keepaspectratio]{Aula5_files/figure-pdf/unnamed-chunk-8-1.pdf}}

\textbf{Leitura.} Se o termo de interação é significativo, o
\textbf{efeito de wt\_kg} difere entre \texttt{auto} e \texttt{manual}.

\begin{center}\rule{0.5\linewidth}{0.5pt}\end{center}

\subsection{\texorpdfstring{Exemplo B (aplicação socioeconómica):
\texttt{swiss}}{Exemplo B (aplicação socioeconómica): swiss}}\label{exemplo-b-aplicauxe7uxe3o-socioeconuxf3mica-swiss}

\begin{Shaded}
\begin{Highlighting}[]
\FunctionTok{data}\NormalTok{(swiss)}
\NormalTok{modS }\OtherTok{\textless{}{-}} \FunctionTok{lm}\NormalTok{(Fertility }\SpecialCharTok{\textasciitilde{}}\NormalTok{ Agriculture }\SpecialCharTok{+}\NormalTok{ Examination }\SpecialCharTok{+}\NormalTok{ Education }\SpecialCharTok{+}\NormalTok{ Catholic }\SpecialCharTok{+}\NormalTok{ Infant.Mortality,}
           \AttributeTok{data =}\NormalTok{ swiss)}
\FunctionTok{summary}\NormalTok{(modS)}
\end{Highlighting}
\end{Shaded}

\begin{verbatim}

Call:
lm(formula = Fertility ~ Agriculture + Examination + Education + 
    Catholic + Infant.Mortality, data = swiss)

Residuals:
     Min       1Q   Median       3Q      Max 
-15.2743  -5.2617   0.5032   4.1198  15.3213 

Coefficients:
                 Estimate Std. Error t value Pr(>|t|)    
(Intercept)      66.91518   10.70604   6.250 1.91e-07 ***
Agriculture      -0.17211    0.07030  -2.448  0.01873 *  
Examination      -0.25801    0.25388  -1.016  0.31546    
Education        -0.87094    0.18303  -4.758 2.43e-05 ***
Catholic          0.10412    0.03526   2.953  0.00519 ** 
Infant.Mortality  1.07705    0.38172   2.822  0.00734 ** 
---
Signif. codes:  0 '***' 0.001 '**' 0.01 '*' 0.05 '.' 0.1 ' ' 1

Residual standard error: 7.165 on 41 degrees of freedom
Multiple R-squared:  0.7067,    Adjusted R-squared:  0.671 
F-statistic: 19.76 on 5 and 41 DF,  p-value: 5.594e-10
\end{verbatim}

\begin{Shaded}
\begin{Highlighting}[]
\FunctionTok{tidy}\NormalTok{(modS, }\AttributeTok{conf.int =} \ConstantTok{TRUE}\NormalTok{)}
\end{Highlighting}
\end{Shaded}

\begin{verbatim}
# A tibble: 6 x 7
  term             estimate std.error statistic     p.value conf.low conf.high
  <chr>               <dbl>     <dbl>     <dbl>       <dbl>    <dbl>     <dbl>
1 (Intercept)        66.9     10.7         6.25 0.000000191  45.3      88.5   
2 Agriculture        -0.172    0.0703     -2.45 0.0187       -0.314    -0.0301
3 Examination        -0.258    0.254      -1.02 0.315        -0.771     0.255 
4 Education          -0.871    0.183      -4.76 0.0000243    -1.24     -0.501 
5 Catholic            0.104    0.0353      2.95 0.00519       0.0329    0.175 
6 Infant.Mortality    1.08     0.382       2.82 0.00734       0.306     1.85  
\end{verbatim}

\textbf{Interpretação.}\\
- Sinal de \texttt{Education} (negativo) sugere menor fertilidade em
regiões com mais educação, \textbf{controlando} restantes.\\
- \texttt{Catholic} positivo: efeito marginal mantendo o resto.
\textbf{Cuidado}: correlação entre preditores muda a interpretação.

\begin{Shaded}
\begin{Highlighting}[]
\FunctionTok{par}\NormalTok{(}\AttributeTok{mfrow=}\FunctionTok{c}\NormalTok{(}\DecValTok{1}\NormalTok{,}\DecValTok{3}\NormalTok{)); }\FunctionTok{plot}\NormalTok{(modS, }\AttributeTok{which=}\DecValTok{1}\NormalTok{); }\FunctionTok{plot}\NormalTok{(modS, }\AttributeTok{which=}\DecValTok{2}\NormalTok{); }\FunctionTok{plot}\NormalTok{(modS, }\AttributeTok{which=}\DecValTok{5}\NormalTok{); }\FunctionTok{par}\NormalTok{(}\AttributeTok{mfrow=}\FunctionTok{c}\NormalTok{(}\DecValTok{1}\NormalTok{,}\DecValTok{1}\NormalTok{))}
\end{Highlighting}
\end{Shaded}

\pandocbounded{\includegraphics[keepaspectratio]{Aula5_files/figure-pdf/unnamed-chunk-10-1.pdf}}

\begin{center}\rule{0.5\linewidth}{0.5pt}\end{center}

\subsection{Coeficientes padronizados}\label{coeficientes-padronizados}

\begin{itemize}
\tightlist
\item
  Coeficientes \textbf{padronizados} permitem comparar magnitudes
  relativas.
\end{itemize}

\begin{Shaded}
\begin{Highlighting}[]
\NormalTok{swiss\_z }\OtherTok{\textless{}{-}}\NormalTok{ swiss }\SpecialCharTok{|\textgreater{}} \FunctionTok{mutate}\NormalTok{(}\FunctionTok{across}\NormalTok{(}\FunctionTok{everything}\NormalTok{(), scale))}
\FunctionTok{coef}\NormalTok{(}\FunctionTok{lm}\NormalTok{(Fertility }\SpecialCharTok{\textasciitilde{}}\NormalTok{ . , }\AttributeTok{data =}\NormalTok{ swiss\_z))}
\end{Highlighting}
\end{Shaded}

\begin{verbatim}
     (Intercept)      Agriculture      Examination        Education 
    5.161631e-16    -3.129213e-01    -1.647782e-01    -6.704008e-01 
        Catholic Infant.Mortality 
    3.476000e-01     2.511360e-01 
\end{verbatim}

\begin{center}\rule{0.5\linewidth}{0.5pt}\end{center}

\subsection{AIC e BIC --- critérios de seleção de
modelos}\label{aic-e-bic-crituxe9rios-de-seleuxe7uxe3o-de-modelos}

\begin{itemize}
\item
  Ideia: equilibrar \textbf{qualidade de ajustamento}
  (log-verosimilhança) e \textbf{complexidade} (nº de parâmetros \(k\)).
\item
  Definições gerais: \[
  \mathrm{AIC} \;=\; -2\log L(\hat\theta) \;+\; 2k,
  \qquad
  \mathrm{BIC} \;=\; -2\log L(\hat\theta) \;+\; k\log n.
   \] \textbf{AIC} (Akaike Information Criterion) privilegia
  \textbf{previsão} (penalização \(2k\));
\end{itemize}

\textbf{BIC} favorece \textbf{parcimónia} (penalização \(k\log n\)).

\begin{center}\rule{0.5\linewidth}{0.5pt}\end{center}

\textbf{Como usar} - \textbf{Quanto menor, melhor} (AIC/BIC em valores
absolutos não têm significado isolado).\\
- Compare \textbf{modelos ajustados aos mesmos dados}; não é preciso
serem aninhados.

\textbf{Notas}

\begin{itemize}
\tightlist
\item
  Preferir \textbf{AIC} se o objetivo principal é \textbf{previsão}.\\
\item
  Preferir \textbf{BIC} se privilegia
  \textbf{simplicidade}/interpretação (penaliza mais o número de
  parâmetros).\\
\item
  Não substitui \textbf{diagnóstico} do modelo (resíduos, colinearidade,
  etc.).
\end{itemize}

\begin{center}\rule{0.5\linewidth}{0.5pt}\end{center}

\begin{Shaded}
\begin{Highlighting}[]
\NormalTok{mods }\OtherTok{\textless{}{-}} \FunctionTok{list}\NormalTok{(}
  \AttributeTok{base  =} \FunctionTok{lm}\NormalTok{(mpg }\SpecialCharTok{\textasciitilde{}} \DecValTok{1}\NormalTok{, }\AttributeTok{data =}\NormalTok{ mt),}
  \AttributeTok{pars  =} \FunctionTok{lm}\NormalTok{(mpg }\SpecialCharTok{\textasciitilde{}}\NormalTok{ wt\_kg }\SpecialCharTok{+}\NormalTok{ hp, }\AttributeTok{data =}\NormalTok{ mt),}
  \AttributeTok{int   =} \FunctionTok{lm}\NormalTok{(mpg }\SpecialCharTok{\textasciitilde{}}\NormalTok{ wt\_kg }\SpecialCharTok{*}\NormalTok{ am }\SpecialCharTok{+}\NormalTok{ hp, }\AttributeTok{data =}\NormalTok{ mt),}
  \AttributeTok{full  =} \FunctionTok{lm}\NormalTok{(mpg }\SpecialCharTok{\textasciitilde{}}\NormalTok{ wt\_kg }\SpecialCharTok{+}\NormalTok{ hp }\SpecialCharTok{+}\NormalTok{ am }\SpecialCharTok{+}\NormalTok{ qsec }\SpecialCharTok{+}\NormalTok{ gear, }\AttributeTok{data =}\NormalTok{ mt)}
\NormalTok{)}
\FunctionTok{tibble}\NormalTok{(}\AttributeTok{modelo =} \FunctionTok{names}\NormalTok{(mods),}
       \AttributeTok{AIC =} \FunctionTok{sapply}\NormalTok{(mods, AIC),}
       \AttributeTok{BIC =} \FunctionTok{sapply}\NormalTok{(mods, BIC))}
\end{Highlighting}
\end{Shaded}

\begin{verbatim}
# A tibble: 4 x 3
  modelo   AIC   BIC
  <chr>  <dbl> <dbl>
1 base    209.  212.
2 pars    157.  163.
3 int     152.  160.
4 full    156.  167.
\end{verbatim}

\begin{center}\rule{0.5\linewidth}{0.5pt}\end{center}

\subsection{Boas práticas}\label{boas-pruxe1ticas}

\begin{enumerate}
\def\labelenumi{\arabic{enumi}.}
\tightlist
\item
  Verificar pressupostos pela análise dos resíduos.
\item
  Verificar \textbf{pontos influentes}.\\
\item
  Causalidade ≠ correlação.\\
\item
  Explicar coeficientes em \textbf{unidades} e \textbf{cenários}
  relevantes.\\
\item
  Quantificar incerteza: \textbf{ICs} e \textbf{intervalos de predição}.
\end{enumerate}

\begin{center}\rule{0.5\linewidth}{0.5pt}\end{center}

\subsection{Exercício}\label{exercuxedcio}

\begin{itemize}
\tightlist
\item
  Ajustar
  \(\text{mpg} \sim \text{wt}_\text{kg} + \text{hp} + \text{am}\) e
  reportar IC(95\%) de cada coeficiente.\\
\item
  Testar a interação \(\text{wt}_\text{kg}:\text{am}\) e interpretar o
  termo cruzado.\\
\item
  Em \texttt{swiss}, calcular VIFs e discutir colinearidade; refazer o
  modelo com variáveis padronizadas.\\
\item
  Construir um cenário de previsão com \textbf{IC da média} e
  \textbf{intervalo de predição}.
\end{itemize}




\end{document}
